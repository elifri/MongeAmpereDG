\documentclass[a4paper,11pt]{article}
\usepackage[T1]{fontenc}
\usepackage[utf8]{inputenc}
\usepackage{lmodern}
\usepackage[T1]{fontenc}
\usepackage[utf8]{inputenc}
\usepackage{lmodern}
\usepackage{ngerman}
\usepackage{cite}
\usepackage{amssymb}
\usepackage{amsfonts}
\usepackage{amsmath}
\usepackage{stmaryrd}

\newcommand{\D}{\operatorname{D}}


\title{Notizen für Finite Elemente Verfahren}
\author{Elisa}


\begin{document}

\maketitle

\section*{C++}

für Randbedingungen $u\mid_{\partial \Omega}=g$ und rechte seite $f$ 
wird das LGS $LM u = Lrhs$ mit
\[
	LM = lhs + Randterme + penalty
\]
Dabei ist in leafcell berechnet
\[
	lhs(\varphi,u) = -\int_\Omega (A \nabla \varphi)^T \cdot \nabla u dx
\]

im assemble\_MA wird berechnet
\begin{align*}
	Randterme(\varphi,u) = \int_{inner \: faces} \lVert \varphi \rVert \{\{A\nabla u \cdot n\}\} dS 
										+ \int_{inner \: faces} \lVert u \rVert \{\{A\nabla \varphi \cdot n\}\} dS \\
										+ \int_{outer \: faces} \varphi \cdot (A\nabla u \cdot n) dS 
										+ \int_{outer \: faces} u \cdot (A\nabla \varphi \cdot n) dS	
\end{align*}

und
\[
	penalty(\varphi,u) = \int_{inner \: faces} \sigma \lVert \varphi \rVert \lVert u \rVert dS
	                  + \int_{outer \: faces} \sigma \varphi u dS
\]

Für die Lrhs wird gerechnet:
\[
	Lrhs(\varphi) = \int_\Omega f \varphi 
	             +\int_{outer \: faces} g A \nabla \varphi 
	             + \sigma \int_{outer \: faces} g \lVert varphi \rVert
\]

Achtung $\sigma$ wird in jeder Iteration mit $(iteration+1)*10$ multipliziert

\end{document}
