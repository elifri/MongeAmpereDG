\section{Benchmark Examples}

Since the \MA equation became a benchmark problem for fully nonlinear second order PDEs there are same classical test problems for the two-dimensional case. All test cases are solved on the unitsquare $\Omega=[0,1]^2$.

\begin{test} \label{test smooth}
The first classical \MA test is the problem with the data
\[
	u=\exp( \lVert x \rVert_2^2  /2) 
	\text { and } 
	f = (1 + \lVert x \rVert_2^2) \exp( \lVert x \rVert^2).
\]
It has a very smooth solution 

\end{test}

\begin{test}\label{test sqrt}
The data
\[
	u = - \sqrt{ 2-  \lVert x \rVert_2^2}
	\text { and } 
	f = 2\left( 2-  \lVert x \rVert_2^2 \right)^{-2}
\]
defines the second example. This test is especially interesting because the convex viscosity solution is only contained in $W^{1,p}(\Omega) $ for $p \in [0,4)$\cite{DG2006a}, i.e. it lacks $H^2$ regularity.
\end{test}

For the next two tests we define $x_0 = \left(\frac 1 2, \frac 1 2  \right)^t$.

\begin{test}\label{test singularity}
The third \MA test is given by
\[
	u=\frac 1 2 \left( \max 0 {\lVert x - x_0 \rVert_2-0.2 }  \right)^2 
	\text { and } 
	f = \max 0 {1-\frac {0.2} {\lVert x - x_0 \rVert_2} }.
\]
\end{test}


\begin{test}\label{test dirac}
A test where the analytical solution is unknown is determined by
\[
	u = \lVert x - x_0 \rVert_2
	\text { and } 
	f = \pi \delta_{x_0}
\]
defines the third example.
\end{test}


\begin{test}\label{test rhsConst}
A test where the analytical solution is unknown is determined by
\[
	u = 0 \text{ on } \partial \Omega
	\text { and } 
	f = 1
\]
defines the third example.
\end{test}


\section{Numerical Results of a Finite Element Method based on a Disrete Hessian}

For reference I implemented the algorithm introduced in Section \ref{sec: FEM discrete Hessian}.
Additional to the numerical results Neilan presented it is interesting to explore what happens if we vary the polynomial degree for the Hessian ansatz space. 

The implementation was done with the Finite Element Tool FEniCS \cite{FEniCS}. The uniform triangulation $\triang$ was obtained by first dividing the domain into squares of side length $h$ and then split them into 4 triangles by drawing both diagonals. \\
Different to the initial guess suggested in \cite{Neilan2014} I used the solution of $\triangle u = -\sqrt{2f}$ as introduced in \ref{sec: initial guess}. 
We denote the degree of the trial space $V_h=P_h^k \cap H^1(\Omega)$ by $k$ and the degree chosen for the Hessian ansatz space $\Sigma_h = [\mathcal{P}_h^{k_{DH}}]^{d \times d}$ by $k_{DH}$. The nonlinear system of equations given by \eqref{eq: neilan eq1} and \eqref{eq: discrete hessian} or \eqref{eq: neilan eq1 + jump} and \eqref{eq: discrete hessian}, respectively was solved with FEniCS' internal Newton solver. Its parameters for absolute and relative tolerance were set to $1e-8$ and the number of maximum iteration was restricted to 100. 

Figure \ref{fig: l2 errors test 1} shows the $L^2$ error of all performances of the first test case that have converged, the polynomial degrees $k$ were taken to be $1,\dots,3$ and $k_{DH}$ equal to all variants $0, \dots, k$.  The results for the runs with $k=3$ are shown more detailed in Table \ref{tab: l2 errors test 1 deg 2}, in both tables the column $N$ refers to the number of iterations the Newton solver needed to reach the desired tolerance. 

%read data for case deg=22 and merge into one file
\newcommand{\readDataN}[2]{
\pgfplotstableread{../../FEniCS/data/#1_l2errornorm} #2

\pgfplotstablecreatecol[copy column from table={../../FEniCS/data/#1_h1errornorm}{h1error}] {h1error} #2

\pgfplotstablecreatecol[copy column from table={../../FEniCS/data/#1_newtonSteps}{steps}] {N} #2
}

%\readDataN{MA1_Brenner_deg2}{\MAOneBrennerTwo}
%\readDataN{MA1_Brenner_deg3}{\MAOneBrennerThree}

%\readDataN{MA3_Brenner_deg2}{\MAThreeBrennerTwo}

%\readDataN{MA4_Brenner_deg2}{\MAFourBrennerTwo}
%\readDataN{MA4_Brenner_deg3}{\MAFourBrennerThree}


\readDataN{MA1_Neilan_GradJump_deg22}{\MAOneJumpdegTwoTwo}
\readDataN{MA1_Neilan_GradJump_deg20}{\MAOneJumpdegTwoZero}

\readDataN{MA1_Neilan_deg33}{\MAOnedegThreeThree}
\readDataN{MA1_Neilan_deg32}{\MAOnedegThreeTwo}

%\readDataN{MA2_Neilan_deg22}{\MATwodegTwoTwo}
%\readDataN{MA2_Neilan_deg33}{\MATwodegThreeThree}

%\readDataN{MA3_Neilan_deg22}{\MAThreedegTwoTwo}
%\readDataN{MA3_Neilan_deg33}{\MAThreedegThreeThree}
\readDataN{MA3_Neilan_GradJump_deg22}{\MAThreeJumpdegTwoTwo}
\readDataN{MA3_Neilan_GradJump_deg33}{\MAThreeJumpdegThreeThree}


%\pgfplotstableread{../../FEniCS/data/MA1_Neilan_deg22_l2errornorm} \MAOnedegTwoTwoL
%\pgfplotstableread{../../FEniCS/data/MA1_NeilanGradJump_deg22_l2errornorm} \MAOneJumpdegTwoTwo
%\pgfplotstableread{../../FEniCS/data/MA1_NeilanGradJump_deg22_h1errornorm}\MAOneJumpdegTwoTwoH

%\pgfplotstablecreatecol[copy column from table={../../FEniCS/data/MA1_NeilanGradJump_deg22_h1errornorm}{h1error}] {h1error} \MAOneJumpdegTwoTwo
%\pgfplotstablecreatecol[copy column from table={../../FEniCS/data/MA1_NeilanGradJump_deg22_newtonsteps}{steps}] {N} \MAOneJumpdegTwoTwo


\begin{figure}[h!]
\centering
	\includegraphics[width = 0.95\textwidth, height = 0.41\textheight]{../../FEniCS/diagrams/MA1_NeilanCG_l2.pdf}
	\caption{$L^2$ errors for test case \ref{test smooth}}
	\label{fig: l2 errors test 1}
\end{figure}

\begin{figure}[h!]
\centering
	\includegraphics[scale =0.5]{../../FEniCS/diagrams/MA1_NeilanCG_h1.pdf}
	\caption{$H^1$ errors for test case \ref{test smooth}}
	\label{fig: h2 errors test 1}
\end{figure}

\begin{table}[h]
	\begin{subtable}[b]{0.45\textwidth}
		\centering
		\pgfplotstabletypeset[columns={iterations, l2error, h1error,N}]\MAOnedegThreeThree
    	\caption{Error for $k=3, k_{DH}=3$}
   \end{subtable}
   ~
	\begin{subtable}[b]{0.45\textwidth}
		\centering
		\pgfplotstabletypeset[columns={iterations, l2error, h1error,N}]\MAOnedegThreeTwo
 	\caption{Error for $k=3, k_{DH}=2$}
	\end{subtable}
	\caption{Errors for test case \ref{test smooth}}
	\label{tab: l2 errors test 1 deg 2}
\end{table}


Considering the results of the smooth test scenario we notice that the error almost do not alter for different polynomial degrees $k_{DH}$ if both converge. Albeit it seems for low polynomial degrees a difference between those degrees leads to divergence.

This test scenario was also performed with the additional normal jump penalty term as stated in \eqref{eq: neilan eq1 + jump} weighted with $\eta$ equal to 50 leading to the results shown in figure \ref{fig: l2 errors test 1 jump} and the tables \ref{tab: l2 errors test 1 deg 2 jump}.

\begin{figure}[h!]
	\includegraphics[scale=0.5]{../../FEniCS/diagrams/MA1_NeilanCG_GradJump_l2.pdf}
	\caption{$L^2$ errors for test case \ref{test smooth} and additional gradient jump penalty}
	\label{fig: l2 errors test 1 jump}
\end{figure}

\begin{figure}[h!]
	\includegraphics[width = 0.95\textwidth, height = 0.41\textheight]{../../FEniCS/diagrams/MA1_NeilanCG_GradJump_h1.pdf}
	\caption{$H^1$ errors for test case \ref{test smooth} and additional gradient jump penalty}
	\label{fig: h1 errors test 1 jump}
\end{figure}

\begin{table}[h]
	\begin{subtable}[b]{0.45\textwidth}
		\centering
		\pgfplotstabletypeset[columns={iterations, l2error, h1error,N}]\MAOneJumpdegTwoTwo
    	\caption{Error for $k=2, k_{DH}=2$}
   \end{subtable}
   ~
	\begin{subtable}[b]{0.45\textwidth}
		\centering
		\pgfplotstabletypeset[columns={iterations, l2error, h1error,N}]\MAOneJumpdegTwoZero
 	\caption{Error for $k=2, k_{DH}=0$}
	\end{subtable}
	\caption{Errors for test case \ref{test smooth}}
	\label{tab: l2 errors test 1 deg 2 jump}
\end{table}

\todo{Kommentar zu diesen Ergebnissen.}

The test two was carried out for $\eta=20k^2$


\newpage

\section{Numerical Results of Our DG Method}

Our method was implemented in pure C++. The underlying grid structure was taken from the implementation of \cite{BMV2009}, handling vectors and matrices as well as was linear solver were provided by the Eigen library \cite{eigenweb}. To solve the quadratic program arising in the context of convexification the C++ library ipopt \cite{ipopt} was used.

For a initial guess I also used the solution of $\triangle u = -\sqrt{2f}$ combined with the multilevel approach as described in \ref{sec: initial guess}.
The arising linear system of equations was solved with Eigen's internal Cholesky solver.
The parameters were if not other specified taken to be $\sigma=blub, \sigma_G = blubberdiblub, \alpha =
0.3$ and $\varepsilon = 1e-2$.
In the first figure \ref{fig: l2 errors test 1 ourMethod} there were carried out 15 steps before a further refinement.

\begin{figure}[h!]
	\includegraphics[scale =0.5]{../../temp.pdf}
	\caption{$L^2$ errors for test case \ref{test smooth} and additional gradient jump penalty}
	\label{fig: l2 errors test 1 ourMethod}
\end{figure}


?????
While Awanou uses a finite difference scheme to solve the PDE in every step we want to exploit the benefits of a DG method and solve every intermediate step by a SIPG method. 
