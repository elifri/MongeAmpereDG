\section{Benchmark Examples}

Since the \MA equation became a benchmark problem for fully nonlinear second order PDEs there are same classical test problems for the two-dimensional case. All test cases are solved on the unitsquare $\Omega=[0,1]^2$.

\begin{test} \label{test smooth}
The first classical \MA test is the problem with the data
\[
	u=\exp( \lVert x \rVert_2^2  /2) 
	\text { and } 
	f = (1 + \lVert x \rVert_2^2) \exp( \lVert x \rVert^2).
\]
It has a very smooth solution 

\end{test}

\begin{test}\label{test sqrt}
The data
\[
	u = - \sqrt{ 2-  \lVert x \rVert_2^2}
	\text { and } 
	f = 2\left( 2-  \lVert x \rVert_2^2 \right)^{-2}
\]
defines the second example. This test is especially interesting because the convex viscosity solution is only contained in $W^{1,p}(\Omega) $ for $p \in [0,4)$\cite{DG2006a}, i.e. it lacks $H^2$ regularity.
\end{test}

For the next two tests we define $x_0 = \left(\frac 1 2, \frac 1 2  \right)^t$.

\begin{test}\label{test singularity}
The third \MA test is given by
\[
	u=\frac 1 2 \left( \max 0 {\lVert x - x_0 \rVert_2-0.2 }  \right)^2 
	\text { and } 
	f = \max 0 {1-\frac {0.2} {\lVert x - x_0 \rVert_2} }.
\]
\end{test}


\begin{test}\label{test dirac}
A test where the analytical solution is unknown is determined by
\[
	u = \lVert x - x_0 \rVert_2
	\text { and } 
	f = \pi \delta_{x_0}
\]
defines the third example.
\end{test}


\begin{test}\label{test rhsConst}
A test where the analytical solution is unknown is determined by
\[
	u = 0 \text{ on } \partial \Omega
	\text { and } 
	f = 1
\]
defines the third example.
\end{test}


\section{Numerical Results of a Finite Element Method based on a Disrete Hessian}

For reference I implemented the algorithm introduced in Section \ref{sec: FEM discrete Hessian}.
Additional to the numerical results Neilan presented it is interesting to explore what happens if we vary the polynomial degree for the Hessian ansatz space. 

The implementation was done with the Finite Element Tool FEniCS \cite{FEniCS}. The uniform triangulation $\triang$ was obtained by first dividing the domain into squares of side length $h$ and then split them into 4 triangles by drawing both diagonals. \\
Different to the initial guess suggested in \cite{Neilan2014} I used the solution of $\triangle u = -\sqrt{2f}$ as introduced in \ref{sec: initial guess}. 
We denote the degree of the trial space $V_h=P_h^k \cap H^1(\Omega)$ by $k$ and the degree chosen for the Hessian ansatz space $\Sigma_h = [\mathcal{P}_h^{k_{DH}}]^{d \times d}$ by $k_{DH}$.
Figure \ref{fig: l2 errors test 1} shows the $L^2$ error calculated for the first test case, the numbers for the cases with $k=2$ are shown more detailed in Table \ref{tab: l2 errors test 1 deg 2}. For a run with $k=3$ and $k_{DH}=0$  the Newton solver diverged already for $h=1/2$. \todo{Kommentar?}

%%read data for case deg=22 and merge into one file
\newcommand{\readDataN}[2]{
\pgfplotstableread{../../FEniCS/data/#1_l2errornorm} #2

\pgfplotstablecreatecol[copy column from table={../../FEniCS/data/#1_h1errornorm}{h1error}] {h1error} #2

\pgfplotstablecreatecol[copy column from table={../../FEniCS/data/#1_newtonSteps}{steps}] {N} #2
}

%\readDataN{MA1_Brenner_deg2}{\MAOneBrennerTwo}
%\readDataN{MA1_Brenner_deg3}{\MAOneBrennerThree}

%\readDataN{MA3_Brenner_deg2}{\MAThreeBrennerTwo}

%\readDataN{MA4_Brenner_deg2}{\MAFourBrennerTwo}
%\readDataN{MA4_Brenner_deg3}{\MAFourBrennerThree}


\readDataN{MA1_Neilan_GradJump_deg22}{\MAOneJumpdegTwoTwo}
\readDataN{MA1_Neilan_GradJump_deg20}{\MAOneJumpdegTwoZero}

\readDataN{MA1_Neilan_deg33}{\MAOnedegThreeThree}
\readDataN{MA1_Neilan_deg32}{\MAOnedegThreeTwo}

%\readDataN{MA2_Neilan_deg22}{\MATwodegTwoTwo}
%\readDataN{MA2_Neilan_deg33}{\MATwodegThreeThree}

%\readDataN{MA3_Neilan_deg22}{\MAThreedegTwoTwo}
%\readDataN{MA3_Neilan_deg33}{\MAThreedegThreeThree}
\readDataN{MA3_Neilan_GradJump_deg22}{\MAThreeJumpdegTwoTwo}
\readDataN{MA3_Neilan_GradJump_deg33}{\MAThreeJumpdegThreeThree}


%\pgfplotstableread{../../FEniCS/data/MA1_Neilan_deg22_l2errornorm} \MAOnedegTwoTwoL
%\pgfplotstableread{../../FEniCS/data/MA1_NeilanGradJump_deg22_l2errornorm} \MAOneJumpdegTwoTwo
%\pgfplotstableread{../../FEniCS/data/MA1_NeilanGradJump_deg22_h1errornorm}\MAOneJumpdegTwoTwoH

%\pgfplotstablecreatecol[copy column from table={../../FEniCS/data/MA1_NeilanGradJump_deg22_h1errornorm}{h1error}] {h1error} \MAOneJumpdegTwoTwo
%\pgfplotstablecreatecol[copy column from table={../../FEniCS/data/MA1_NeilanGradJump_deg22_newtonsteps}{steps}] {N} \MAOneJumpdegTwoTwo


%\begin{table}[h]
%	\begin{subtable}[b]{0.45\textwidth}
%		\centering
%		\pgfplotstabletypeset[columns={iterations, l2error, h1error}]\MAOneJumpdegTwoTwo
 %   	\caption{Error for $k=2, k_{DH}=2$}
 %   \end{subtable}
  %  ~
%	\begin{subtable}[b]{0.45\textwidth}
%		\centering
%		\pgfplotstabletypeset[columns={iterations, l2error, h1error}]\MAOneJumpdegTwoZero
 % 	\caption{Error for $k=2, k_{DH}=0$}
%	\end{subtable}
%	\caption{Errors for test case \ref{test smooth}}
%	\label{tab: l2 errors test 1 deg 2}
%\end{table}

deg =3 $deg_h$essian =0 funktioniert nicht.

\begin{figure}[h]
	\includegraphics[scale=0.5]{../../FEniCS/diagrams/MA1_NeilanGradJump_l2.pdf}
	\caption{$L^2$ errors for test case \ref{test smooth}}
	\label{fig: l2 errors test 1}
\end{figure}

\section{Numerical Results of Our DG Method}


In \cite{Awanou2014} Awanou analysed the similar iteration process
\begin{align}
	\nabla \cdot \left( \cof(D^2 u^0) \nabla u^{i+1} \right) &= \nabla \cdot \left( \cof(D^2 u^0) \nabla u^{i} \right) + f - \operatorname{det} (D^2u^i) \textnormal{ in } \Omega,  \label{eq: Awanout eq}\\
	u^{i+1} &= g \textnormal{ on } \partial \Omega 
\end{align}
showing convergence for the analytical solution $u$ and a sufficent close $u^0$. 
In a earlier work \cite{Awanou2010} Awanou examined a discrete Version of a vanishing moment method, herein he mentioned Newton's method defined by
\[
	\int_{\Omega} [\mycof{ D^2 u_h^i} Du_h^{i+1}] \cdot Dv_h = -	\int_{\Omega} f v_h + \frac 1 2 \int_{\Omega} [\mycof{ D^2 u_h^i} Du_h^{i}] \cdot Dv_h \; \forall v_h \in V_h \cap H^1_0 (\Omega)  \label{eq: Awanout eq2}.
\]
He claims that this ansatz for problems with non-smooth solutions breaks down, in his numerical results he cites test \ref{test sqrt} as an example where Newton's method diverges.

?????
While Awanou uses a finite difference scheme to solve the PDE in every step we want to exploit the benefits of a DG method and solve every intermediate step by a SIPG method. 
