This chapter covers the main results on the \MA equation. After a definition of our main PDE we very briefly discuss solution concepts, as well as existence and uniqueness results. Thereafter we derive a linearisation of the \MA operator and conclude with a short overview of existing numerical schemes solving \MA equations. 


\section{The \MA Equation}

The \MA Equation with Dirichlet boundary data states
\begin{align}
	 \mydet{D^2 u} &= f \textnormal{ in } \Omega \label{MA eq}\\
	 u & = g \textnormal{ on } \partial \Omega,
\end{align}
where $D^2 u$ denotes the Hessian of $u$.
It is a general fully nonlinear partial differential equation meaning it is nonlinear in the highest order derivatives, namely the second derivatives.

\begin{proposition}
	The \MA equation is elliptic for strictly convex function $u$ and $f > 0$.
\end{proposition}
For a proof of this result we refer to \cite{CC1995, GT1977}. In other words only restricted to convex solutions and for a positive right-hand side the \MA operator is elliptic.

Note, different from the well-known linear case we cannot derive a weak formulation based on weak derivatives/integration by parts. Because of the nonlinearity in the highest derivatives we are not able to shift derivatives to test functions. To the author's knowledge there is no general variational or weak formulation for fully nonlinear PDEs \cite{FGN2013}.
However, there are attempts to establish less regular weak solutions to the \MA equation:

\section{Weak Solution Concepts}
First Crandall and Lions \cite{CL1983} introduced the notion of \emph{viscosity solutions} for fully nonlinear first order PDEs. The developed theory was later extended to also handle second order PDEs.
\begin{definition}[Viscosity Solution{\cite[Definition 1.1]{FGN2013}}]
	Let $F$ be an elliptic  fully nonlinear second order operator.
	$u \in C^0(\Omega)$ is called a \emph{viscosity subsolution (supersolution)} of $F(D^2u(x), \nabla u(x), u(x), x)=0$  if, for every point $x_0 \in \Omega$ and function $\varphi \in C^2(\Omega)$ satisfying $u \leq \varphi (\geq \varphi)$ in $\Omega$ and $u(x_0) = \varphi(x_0)$, there holds $\mydet{D^2 \varphi(x_0)} - f \leq 0 (\geq 0)$. 
%	\[
%	\forall x_0 \in \Omega \forall \varphi \in C^2(\Omega), u \leq \varphi (\geq \varphi)
%	\] holds
%	\[
%	u(x_0) = \varphi(x_0) : F[\varphi](x_0) \leq 0 (\geq 0)
%	\]
	$u$ is called a \emph{viscosity solution} if $u$ is simultaneously a viscosity sub- and supersolution.
\end{definition}

We see directly that a convex classical solution is also a viscosity solution. Conversely, viscosity solutions are solutions in a weaker sense and there are examples where only a viscosity solution exists, see for example a later Benchmark test \ref{test sqrt}.
It proved very hard to develop reliable numerical schemes converging to the viscosity solution and even harder to come up with proves of convergence to the viscosity solution.

%\subsection{Aleksandrov Solution}
A different kind of solution concept is derived by reformulating the \MA problem in two dimensions via a differential geometry interpretation.

\begin{definition}[\MA Measure {\cite[2.1.1.]{FGN2013}}]\label{def:MA measure}
	The \emph{\MA measure} associated with $u$ is defined as 
	\begin{align}
		\mathcal{M}_u (E) = \mathcal{L}^2(\partial u(E)) \qquad \text{ for any Borel set } E \subset \Omega,
	\end{align}
	where $\mathcal{L}^2$ denotes the two-dimensional Lebesgue measure and 
	\[
		\partial u(E) = \bigcup_{x \in E} \partial u (x) = \bigcup_{x \in E} \{p \in \R^2; u(y) \geq u(x) +p\cdot (y-x) \forall y \in \Omega\}
	\]
	denotes the union of the \emph{subdifferentials (or normal mappings)} of $u$ with respect to $E$.
\end{definition}
The corresponding \MA problem is then defined by

\begin{definition}[Aleksandrov Solution {\cite[2.1.1.]{FGN2013}}]\label{def:aleksandrov solution}
A convex function $u$ such that for a given Radon measure $\mu$ it fulfills
\begin{align}
\begin{split}
\mathcal M_u&= \mu \\ 
u &= g  \text{ on } \partial \Omega
\end{split}
\end{align}
is called \emph{Aleksandrov solution} of the \MA equation.
\end{definition}

Aleksandrov solution are also viscosity solutions if $\mu$ is absolutely continuous with Lebesgue measure with a continuous density $f$ \cite[proposition 1.3.4.]{Gutierrez2001} . For $f > 0$ the two solution concepts even coincide \cite[proposition 1.7.1]{Gutierrez2001}. 

\section{Existence and Uniqueness}
Showing existence of fully nonlinear PDEs is a hard challenge compared to the linear case. Nevertheless it is an important part of the well-posedness of a numerical problem. 
For simplicity we constraint ourselves to the elliptic case. 

The existence of classical solutions hangs on the regularity of the right-hand side $f$ and boundary data $g$, as well as on the problem domain $\Omega$. There are several variants and relaxations on assumptions made on $\Omega,f$ and $g$, mostly $\Omega$ is taken to be convex with a smooth boundary and $f,g$ to be smooth, results can be found e.g. in \cite{Gutierrez2001, GT1983, Urbas1998}.

The \MA equation is known for its notorious ambiguity: If $\Omega$ is a domain, one can proof it has at most two solutions, compare \cite[Kap.IV, \S 5,3]{CH1968}. In general these two solutions also exist.
A finite difference method with a standard nine-point stencil and Newton's method to solve the resulting nonlinear system produce fore different initial guesses even sixteen different numerical solutions on a $4 \times 4$ grid\cite{FGN2013}. 

In 1989 Ishii proved existence of viscosity solution using the comparison principle and Perron's method \cite{Ishii1989}. 
A main thesis for uniqueness theory is the comparison principle for second order equations.

\begin{theorem}[Comparison Principle]
	Let $F$ be a fully nonlinear second order operator continuous in all of its arguments.
	If $F$ is elliptic,	u,v are, respectively a viscosity subsolution and a viscosity supersolution to $F[u](x)=0$ and $u \leq v$ on $\partial \Omega$, then $u \leq v$ in all of $\bar \Omega$.
\end{theorem}
A proof can be found for example in \cite[Theorem 17.1]{GT1983}

A viscosity solution is unique by the comparison principle. If $u$ and $v$ are two viscosity solutions of $\mydet{D^2 u}=f$ with $u=v$ on $\partial \Omega$, then especially holds $u \leq v$ on $\partial \Omega$ and hence by the comparison principle $u \leq v$ in  $\bar \Omega$. Applying the same argument for $v \leq u$ on $\partial \Omega$ we find equality of the solutions. Note, that a viscosity has to be convex since otherwise the \MA operator is not elliptic, hence  

\section{The Linearisation of the \MA operator}

	\begin{theorem}[Linearisation] \label{thm: linearisation}
		The Linearisation of the \MA operator $F(u)=\mydet{D^2u}-f$ is given by
		\begin{align}
			DF[u]v = \nabla \cdot \left( \mycof {D^2 u } \nabla (v) \right) \text{ or } DF[u]v=\mycof{D^2 u}:D^2 (v)		\end{align}
	where $DG[u]$ denotes the \emph{G\^ateaux derivative} of $G$ given by
	\[
		DG[u]v = \lim\limits_{\varepsilon \rightarrow 0} \frac { G(u+\varepsilon v) - G(u)}\varepsilon.
	\]
	\end{theorem}
		
	\begin{proof}
	We define the \MA bilinear form by
	\[
	\bilin v w = \mycof{D^2 v}:D^2w.
	\]
	Note that by Lemma \ref{la: rel det cofactor} then $\mydet {D^2 u} = \frac 1 2 \bilin u u$ holds.
	
	Since the \MA form is bilinear and symmetric we have the identity
	\begin{align}
		\mydet{D^2 (v+w)} =& \frac 1 2 \bilin{v+w}{v+w} = \frac 1 2 \left(\bilin v v + 2 \bilin v w + \bilin w w \right)  \\
		=&  \detHess v  + \cofHess v : D^2 w + \detHess w \label{eq: det addition}
	\end{align}
	 Let us now look at the G\^ateaux derivative of $F$
		\begin{align*}
			DF[u]v =& \lim\limits_{\varepsilon \rightarrow 0} \frac { \mydet{D^2(u+\varepsilon v)} - \mydet{D^2u}}\varepsilon\\
			\end{align*}
Applying \eqref{eq: det addition} we find
		\begin{align*}
			DF[u]v =& \lim\limits_{\varepsilon \rightarrow 0} 
										\frac  {\detHess u + \cofHess u : D^2(\varepsilon v) + \detHess {\varepsilon v} - \detHess u}
													\varepsilon\\ 
			 =& \lim\limits_{\varepsilon \rightarrow 0} 
										\frac  {\varepsilon \cofHess u : D^2(v) + \varepsilon^d \detHess v}
													\varepsilon\\ 
			\stackrel{d > 1} =& \mycof{D^2 u}:D^2v
		\end{align*}
Considering also the result of Lemma \ref{la: An application of the divergernce product rule} the claim follows.
	\end{proof}	


\section{Types of Numerical Methods for the \MA Equation}
Since equations of \MA type arise in many areas such as optics, transportation etc. there has been an active research on how to solve. Except for some simple special cases there is no analytical solution known resulting in a widely varied numerical set of methods.

They can roughly be divided in four categories: Direct Finite difference schemes, methods based on variational principles and approximating infinite-dimensional spaces by finite-dimensional spaces, methods based on finite basis expansions and approximating PDEs at sampling points and the rest. \\
In the last category one can mainly find methods as the lattice Boltzmann method, these are often designed for special applications.

A lot of the methods result in hard to handle nonlinear system. The most favoured method to solve this system is Newton's method // Often the method of choice is Newton's method. However, it requires an appropriate initial guess//accurate starting value. Finding//providing those is sometimes as tricky//hard as the problem itself.

We give now an short overview of the different methods and some of their most known representatives. For a more detailed survey we refer the interested reader to the overview article \cite[Section 2.1]{FGN2013}.


\subsection{Direct Finite Difference Schemes}

We recall the first category consists of direct finite difference schemes, where the term Direct refers to the fact that these schemes directly approximate the in \eqref{MA eq} occurring derivatives. 
There are two points in favor of those schemes. First, because of their simplicity methods finite difference schemes widely used in "application areas".  And second Barles and Souganidis set up a general framework to analyse finite difference schemes for fully nonlinear elliptic PDEs satisfying a comparison principle \cite{BS1991}, this work especially reduces the convergence analysis for monotone schemes to the verification of stability and consistency.
%As one desires monotone schemes for there exist theorems on their convergence, compare \cite[Theorem 2.1]{BS1991} and monotone schemes even for linear PDEs require wide stencil, compare \cite{MW1953} many finite difference schemes are wide-stencil schemes. 
Built on this structure and are the works of Benamou, Oberman and Froese \cite{BFO2010, Oberman2008, FO2011}, where they provide some monotone wide-stencil for the \MA equation.
They are able to prove convergence to the unique convex viscosity solution and experiments show the mentioned methods turn out to be very robust. To the author's knowledge until today all methods proven to converge to the viscosity solution are finite difference schemes.
Nevertheless these methods have a huge drawback. Monotone schemes even for linear PDEs require wide stencils, compare \cite{MW1953}  and in order to increase accuracy one has not only to refine the grid, but also to increase the size of the stencil. These stencils make the methods difficult to apply if the problem domain $\Omega$ is not easily approximated by rectangles. Another disadvantage is that in general there are no rate of convergence is given, even for classical solutions.

\subsection{Methods based on Variational Principles and Approximating Infinite-dimensional Spaces by Finite-dimensional Spaces}
 We concentrate ourselves in this work to methods of the second area to which beside DG methods methods such as finite element methods, augmented Lagrangian methods, least squares methods and boundary element methods belong. Current DG methods we discuss in the next chapter \ref{ch:DGMongeAmpere} in detail.
To make use of least square methods one needs at first to formulate the \MA problem as a constrained minimalisation problem.
\begin{definition}[Constrained Minimalisation Formulation, \cite{FGN2013}]
Find $(u,p) \in \mathcal{S}$ such that
\begin{align}
	j(u,p) \leq& j(v,q) \qquad \forall \{v,q\} \in \mathcal{S}, \text{ where }\\
	j(v,q)  =& \frac 1 2 \int_{\Omega} |\triangle v |^2, \\
	\mathcal{S} =& \{(v,q)\in V_g \times Q^+_f, D^2v-q =0\},\\
	V_g =& \{v \in H^2(\Omega), v \arrowvert_{\delta \Omega}=g\} \\
	Q^+_f =& \{q \in [L^2(\Omega)]^{2\times2} , q_{12} =q_{21},  q_{11} > 0,q_{22} > 0, \mydet q = f\}.
\end{align}

\end{definition}
%Ensuing from this formulation Dean and Glowinski formulated a saddle point with an augmented Lagrangian and provided an algorithm computing its solution in \cite{DG2004}. The algorithm splits the nonlinear \MA problem to 
Based on this formulation Dean and Glowinski developed augmented Lagrangian methods and least square approaches \cite{DG2004,DG2006,DG2006a}.

Another approach in this category are the so called \emph{vanishing moment methods}. First introduced in \cite{FN2009} they try to transfer the idea of the vanishing viscosity method: Feng and Neilan add higher order terms to obtain a higher order PDE that is easier to handle than the original one. Then the contribution of this higher order terms is decreased approximating thereby the \MA solution. \\
One choice for those higher order terms is the square of the Laplacian of $u$ such that the new PDE is fourth order but linear. Additionally to more terms the newly derived PDE gets further boundary conditions to gain well-posedness , yielding to the problem
\begin{definition}[Vanishing Moment Equations]
	The vanishing moment method approximates \eqref{MA eq} by the sequence
	\begin{align}
		\varepsilon \triangle^2 u^\varepsilon - \mydet{D^2 u^\varepsilon} =& -f \text{ in } \Omega \label{eq: VMM 1}\\ 
		u^\varepsilon =& g \text{ on } \partial \Omega \label{eq: VMM 2}\\
		\triangle u^\varepsilon = & \varepsilon \text{ on } \partial \Omega \label{eq: VMM 3}
	\end{align}
The limit $\lim\limits_{\varepsilon \rightarrow 0 } u^\varepsilon$, if exists is called a \emph{moment solution} of the \MA equation.
Of course the vanishing moments is not directly connected to a solution method, for example one could solve \eqref{eq: VMM 1}-\eqref{eq: VMM 3} with a finite difference scheme or a appropriate Galerkin Method.
Feng and Neilan provide a theoretical analysis for the radially symmetric case and numerical experiments in \cite{FN2009, Neilan2010, FN2011a}. One interesting point is that this method autonomously chooses the unique convex viscosity solution, in even dimensions substituting $\varepsilon$ by $-\varepsilon$ in  \eqref{eq: VMM 1} yields to the other, concave solution to the \MA problem.  
\end{definition}


\subsection{Methods based on Finite Basis Expansions and Approximating PDEs at Sampling Points}
The third category consists of collocation methods and meshless methods.
An example for a spline collocation method is introduced in \cite{BHP2014}. 
Liu and He developed an adaption of an approach of Benamou, Froese and Oberman into to meshless method \cite{LH2013}.
