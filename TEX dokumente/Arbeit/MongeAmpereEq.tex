\section{Notes}
finite Differenzen brauchen verrückte Stencils, zitat von feng [139] einfuegen - Probleme am Rand, Anforderungen an das Gebiet, es gibt keine Beweise zu Konvergenzordnungen
basieren immer auf einem maximumsprinzip

Richardson type Picard Iterationen werden bisher nur auf quasilineare Gleichungen angewandt

\todo{ convex definieren}


\section{The \MA Equation}

The \MA Equation with Dirichlet boundary data states
\begin{align}
	 \mydet{D^2 u} &= f \textnormal{ in } \Omega \label{MA eq}\\
	 u & = g \textnormal{ on } \partial \Omega,
\end{align}
where $D^2 u$ denotes the Hessian of $u$.
It is a general fully nonlinear partial differential equation meaning it is nonlinear in the highest order derivatives, namely the second derivatives.

Note, different from the well-known linear case we cannot derive a weak formulation based on weak derivatives/integration by parts. Because of the nonlinearity in the highest derivatives we are not able to shift derivatives to test functions. Until now there is no general variational/weak formulation for fully nonlinear PDEs \cite{FGN2013}.
However, there are attempts to establish less regular weak solutions to the \MA equation:

%\subsection{Viscosity Solution}
First Crandall and Lions \cite{58} introduced the notion of \emph{viscosity solutions}
\begin{definition}
	$u \in C^0(\Omega)$ is called a \emph{viscosity subsolution (supersolution)} of \eqref{MA eq} if, for every point $x \in \Omega$ and function $\varphi \in C^2(\Omega)$ satisfying $u \leq \varphi (\geq \varphi)$ in $\Omega$ and $u(x) = \varphi(x)$, there holds $\mydet{D^2 \varphi(x)} - f \leq 0 (\geq 0)$. 
	\[
	\forall x_0 \in \Omega \forall \varphi \in C^2(\Omega), u \leq \varphi (\geq \varphi)
	\] holds
	\[
	u(x_0) = \varphi(x_0) : F[\varphi](x_0) \leq 0 (\geq 0)
	\]
	$u$ is called a \emph{viscosity solution} if $u$ is simultaneously a viscosity sub- and supersolution.
\end{definition}

Hier kommt dann eine anschauliche Erklärung zur geometrischen Bedeutung.

%\subsection{Aleksandrov Solution}
The \MA problem in two dimensions can be reformulated via a differential geometry interpretation leading to a different kind of solution concept.

\begin{definition}[\MA Measure \cite{Feng2013}]\label{def:MA measure}
	The \emph{\MA measure} associated with $u$ is defined as 
	\begin{align}
		\mathcal{M}_u (E) = \mathcal{L}^2(\partial u(E)) \qquad \text{ for any Borel set } E \subset \Omega,
	\end{align}
	where $\mathcal{L}^2$ denotes the two-dimensional Lebesgue measure and 
	\[
		\partial u(E) = \bigcup_{x \in E} \partial u (x) = \bigcup_{x \in E} \{p \in \R^2; u(y) \geq u(x) +p\cdot (y-x) \forall y \in \Omega\}
	\]
	denotes the union of the \emph{subdifferentials (or normal mappings)} of $u$ with respect to $E$.
\end{definition}
The corresponding \MA problem is then defined by


\begin{definition}[Aleksandrov Solution 2.1.1. \cite{Feng2013}]\label{def:aleksandrov solution}
A convex function $u$ such that for a given Radon measure $\mu$ it fulfills
\begin{align}
\begin{split}
\mathcal M_u&= \mu \\ 
u &= g  \text{ on } \partial \Omega
\end{split}
\end{align}
is called \emph{Aleksandrov solution} of the \MA equation.
\end{definition}

Aleksandrov solution are also viscosity solutions if $\mu$ is absolutely continuous with Lebesgue measure with a continuous density $f$ \cite{G2001} . For $f > 0$ the two solution concepts even coincide \cite{G2001}. \todo{include proposition 1.3.4. and proposition 1.7.1}

\section{Existence and Uniqueness}
Showing existence of fully nonlinear PDEs is a hard challenge compared to the linear case. For simplicity we constraint ourselves to the elliptic case. 

The \MA equation is known for its notorious ambiguity:
\begin{example}
\todo{here example mit 16 loesungen}
\end{example}

Brenner\cite{Brenner2012} : $\Omega$ smooth $\Rightarrow u$ smooth \cite{Caffarelli1984} und aus $\Omega$ smooth auch $\Rightarrow$ es exisitieren genau 2 lösungen: eine convex, die andere konkav \cite{Courant1989}

A viscosity solution is unique if $F$ is elliptic by the comparison principle.


\section{Types of Numerical Methods for the \MA Equation}
Since equations of \MA type arise in many areas such as optics, transportation etc. there has been an active research on how to solve. Except for some simple special cases there is no analytical solution known resulting in a widely varied numerical set of numerical methods.

They can roughly be divided in four categories: Finite difference schemes, methods based on variational principles and approximating infinite-dimensional spaces by finite-dimensional spaces, methods based on finite basis expansions and approximating PDEs at sampling points and the rest.
because of their simplicity methods of the first category are widely used in "application areas". We concentrate ourselves in this work to methods of the second area to which beside DG methods methods such as finite element methods, least squares methods and boundary element methods belong.
The third category consists of collocation methods and meshless methods.
In the last category one can mainly find methods as the lattice Boltzmann method, these are often designed for special applications.

A lot of the methods result in hard to handle nonlinear system. The most favoured method to solve this system is Newton's method // Often the method of choice is Newton's method. However, it requires an appropriate initial guess//accurate starting value. Finding//providing those is sometimes as tricky//hard as the problem itself.
