This thesis discusses \emph{discontinuous Galerkin} (DG) methods for the \MA equation. 
The \MA equation is the best-known fully nonlinear \emph{partial differential equation} (PDE). It suffices as a prototype problem for numerical schemes challenging fully nonlinear problems and thus enjoys great popularity among the nonlinear PDE community.

Next to its purely mathematical applications equations of \MA type arise in many other fields such as astrophysics, computer science, differential geometry, image processing, finance, optimal transportation and optics\cite{FGN2013}. 
For example the surface of a lens projecting a desired illumination or even image can be described by a \MA equation\cite{KW2010, BHP2014}. \\
These applications demand for efficient and reliable numerical schemes and thus, a lot of methods handling equations of \MA type have been introduced in the last years. Yet a lot of them base on finite difference schemes, Galerkin methods have only been developed recently \cite{Boehmer2008, FN2009a, Awanou2010, BGN+2011, Neilan2014}. DG methods yield more flexibility than other methods as finite difference schemes, especially when it comes to discretise domains. In the optics example the solution describes a lens such that the problem domain is round which is really difficult for finite difference schemes to capture.

\section*{Outline}
The thesis is organised as follows: In chapter \ref{ch:TheoreticalBackground} we refresh the mathematical foundation of discontinuous Galerkin methods and clarify the notation used in the rest of the paper.
At the beginning of the second chapter we state briefly the main results one has to consider when handling \MA equations. In its second part we take a look at present numerical schemes for the \MA equations. As we put particular emphasis on DG methods Chapter \ref{ch:DGMongeAmpere} is about the state of the art for DG methods solving the \MA equation.
Chapter \ref{ch:ourMethod} introduces a new DG method based on a Picard type iteration. In chapter \ref{ch:NumericalResults} we provide a variety of numerical results on a selection of the mentioned DG methods.
Afterwards we conclude the work with a conclusion and perspective.