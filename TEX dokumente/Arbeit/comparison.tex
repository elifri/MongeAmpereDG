\section{Comparison with Newton's method}

Let us apply Newton's method directly on the analytical form of the \MA equation (cf. \eqref{eq: MA eq}).
Let $F$ be the \MA operator, i.e. 
\[
	F(u) = \mydet{D^2 u} -f.
\]
To apply Newton's method on $F(u) =0$ we have to find $u^{i+1}$ satisfying
\begin{align}
	DF[u^i](u^{i+1}-u^i) = -F[u^i],
\end{align}
where $DF[u]$ denotes the Fr\'echet derivative of $F$ at $u$. We derived the Fr\'echet derivative $DF[u] v = \mycof{D^2 u}:D^2v$ already in Theorem \ref{thm: linearisation}, leading us to the Newton iteration
\begin{align}
	\mycof{D^2 u^i}:D^2\left(u^{i+1}-u^i\right) &= -\mydet{D^2 u^i}+f \nonumber \\
	\Leftrightarrow \qquad \qquad  \mycof{D^2 u^i}:D^2(u^{i+1}) &= -\mydet{D^2 u^i} +f  +\mycof{D^2 u^i}:D^2(u^i). \label{eq: Newton iteration pre}
\end{align}

For further rearrangements we recall the identities from Lemma \ref{la: rel det cofactor} and Lemma \ref{la: An application of the divergernce product rule}, i.e. 
\begin{align}
	\nabla \cdot \left( \mycof {D^2 u } \nabla v \right)
	\stackrel{La.\ref{la: An application of the divergernce product rule}}=\mycof{D^2 u}:D^2u
	\stackrel{La.\ref{la: rel det cofactor}}=2 \mydet{D^2u}. \label{eq: cof identities}
\end{align}

Similar to the derivation of the Picard iteration in Section \ref{sec: motivation picard iteration} we can apply \eqref{eq: cof identities} to rewrite \eqref{eq: Newton iteration pre} and we have the problem
\begin{align}
	\nabla \cdot \left( \cof(D^2 u^i) \nabla u^{i+1} \right) &= -\mydet {D^2u^i} +f+\mycof{D^2 u^i}:D^2(u^i)  \textnormal{ in } \Omega,  \label{eq: Newton iteration}\\
	u^{i+1} &= g \textnormal{ on } \partial \Omega .
\end{align}

Hence, considering the fact of \eqref{eq: cof identities} and $\detHess {u^i} \approx f$ we can interpret the Picard method as a variant of Newton's method. 

In \cite{Awanou2014} Awanou analysed an iteration process similar to just presented Newton iteration:
\begin{align}
	\nabla \cdot \left( \cof(D^2 u^0) \nabla u^{i+1} \right) &= \nabla \cdot \left( \cof(D^2 u^0) \nabla u^{i} \right) + f - \operatorname{det} (D^2u^i) \textnormal{ in } \Omega,  \label{eq: Awanout eq}\\
	u^{i+1} &= g \textnormal{ on } \partial \Omega.
\end{align}
Note that term $\mycof{D^2 u^i}:D^2(u^i)$ in \eqref{eq: Newton iteration} equals to $\nabla \cdot \left( \cof(D^2 u^i) \nabla u^{i} \right)$ by \eqref{eq: cof identities}. Thus, Awanou's formulation is the Newton iteration with fixed diffusion matrices.\\
For this analytical formulation he proves convergence to the \MA solution $u$ under the assumption that the initial guess $u^0$ is sufficient close to $u$. Unfortunately, his analysis is not applicable to the Picard iteration as he relies on the fact that the left-hand side of \eqref{eq: Awanout eq} does not alter for different $u^i$.\\
In an earlier work \cite{Awanou2010} Awanou examined a discrete version of a vanishing moment method, herein he mentioned a method he calls Newton's method \cite[(1.3)]{Awanou2010}: Find $u_h^{i+1}\in V_h$ such that for all $v_h \in V_h \cap H^1_0 (\Omega)$
\[
	\myIntX {\Omega} {[\mycof{ D^2 u_h^i} Du_h^{i+1}] \cdot Dv_h} 
	= -	\myIntX{\Omega} {f v_h} 
		+ \frac 1 2 \myIntX{\Omega} {[\mycof{ D^2 u_h^i} Du_h^{i}] \cdot Dv_h} \;  \label{eq: Awanout eq2}.
\]
And indeed this is the variational form of \eqref{eq: Newton iteration} using the identity in \eqref{eq: cof identities} and integration by parts.
He chooses the trial space to consist of piecewise polynomials contained in $C^1(\Omega)$ and claims without further explanation that this ansatz breaks down for problems with non-smooth solutions\cite[Introduction]{Awanou2010}, in his numerical results he cites test \ref{test sqrt} as an example where Newton's method diverges \cite[Section 4.4]{Awanou2010}.

%\begin{align}
%	&\myInt_\Omega \cofHess {u^i} \nabla (u^{i+1}-u^i) \nabla v - \sum_{e \in \edgesb} \myIntS e { \cofHess {u^i} \nabla (u^{i+1}-u^i) \cdot \mathbf n \; v \nonumber \\
%	&- \sum_{e \in \edgesb} \myIntS e { \cofHess {u^i} \nabla v \cdot \mathbf n (u^{i+1}-u^i) + \sigma \sum_{e \in \edgesb} \frac 1 {|e|} \myIntS e { v (u^{i+1}-u^i)\nonumber \\
%	=&-\myInt_\Omega \left(f - \detHess{u^i)}\right) v  
%			- \sum_{e \in \edgesi} \myIntS e { \jump { \average{\cofHess {u^i}} \nabla {u^i} } v \nonumber \\
%&	- \sum_{e \in \edgesb} \myIntS e { \cofHess {u^i} \nabla v \cdot \mathbf n \; (u^i-g) - \sigma \sum_{e \in \edgesb} \frac 1 {|e|} \myIntS e { v (u^{i}-g) \label{eq: a newton step Brenner}
%\end{align}
%To compare both linearisations we first reorder and remove cancelling terms in \eqref{eq: a newton step Brenner}.
%\begin{align}
%	&\myInt_\Omega \cofHess {u^i} \nabla u^{i+1} \nabla v \\
%	&- \sum_{e \in \edgesb} \myIntS e { \cofHess {u^i} \nabla u^{i+1} \cdot \mathbf n \; v 
%		- \sum_{e \in \edgesb} \myIntS e { \cofHess {u^i} \nabla v \cdot \mathbf n \; u^{i+1} \\
%	&+\sigma \sum_{e \in \edgesb} \frac 1 {|e|} \myIntS e { v u^{i+1}\\
%	=
%	&-\myInt_\Omega \left(f - \detHess{u^i)}\right) v \\
%	&- \sum_{e \in \edgesi} \myIntS e { \jump { \average{\cofHess {u^i}} \nabla {u^i} } v 
%		+ \sum_{e \in \edgesb} \myIntS e { \cofHess {u^i} \nabla {u^i} \cdot \mathbf n \; v\\
%	&- 2\sum_{e \in \edgesb} \myIntS e { \cofHess {u^i} \nabla v \cdot \mathbf n \; {u^i}
%		+ \sum_{e \in \edgesb} \myIntS e { \cofHess {u^i} \nabla v \cdot \mathbf n \; g \\
%	&+\sigma \sum_{e \in \edgesb} \frac 1 {|e|} \myIntS e { v g \label{eq: ordered newton step Brenner}
%\end{align}

\section{Analysis of the DG Method} \label{sec: DG analysis}
This section compares the newly derived method with the $C^0$ penalty method by Brenner et al. \cite{BGN+2011}, already presented in Section \ref{sec: Brenner method}, on a analytical basis. In particular we examine how we can transfer the analysis of the penalty method to the Picard iteration.

\subsection{Comparison with a $C^0$ Penalty method}\label{subsec: comparison brenner}
Brenner et al. emphasis that it is important that the linearisation of a \MA method is a consistent and stable discretisation of the linearisation of the \MA equation, namely $\nabla \cdot \left( \mycof {D^2 u } \nabla (v) \right)$ \cite[Remark 2.2]{BGN+2011}.
Therefore we first compare the linearisation of the Picard iteration with the linearisation of the method introduced by Brenner et al. \cite{BGN+2011}.

Let us consider again the linearisation of the finite element method mentioned in Section \ref{sec: Brenner method}, namely the mapping $L^B_u:H^2(\Omega; \triang) \rightarrow [H^2(\Omega; \triang) \cap H^1(\Omega)]'$ (cf. \eqref{eq: var of linearisation}) given by
\begin{align}
\bilin {L^B_{u} w} v
	=&\myIntX  \Omega { (\cofHess{ u}\nabla w) \cdot \nabla v}
		- \sum_{e \in \edgesb} \myIntS e { (\cofHess{u}\nabla w) \cdot \mathbf n \; v} \nonumber \\
		&-  \sum_{e \in \edgesb} \myIntS e { (\cofHess{u}\nabla v) \cdot \mathbf n \; w} 
		+ \sigma \sum_{e \in \edgesb} \frac 1 {|e|} \myIntS e { v w}.
		\label{eq: linearisation brenner}
\end{align}


Looking at the Picard iteration in \eqref{eq:fixed point iteration} we see that we aim to solve the discretised linearisation of the \MA equation at the point $u^i$ in every step.\\
Let $V$ be $H^2(\Omega; \triang) \cap L^2(\Omega)$ and $L_u: V \rightarrow V'$ denote the linear mapping describing the left-hand side of the SIPG method (cf. \eqref{eq: sipg iteration}) to perform an iteration step, i.e.
\begin{align}
	\bilin {L_u w} v =
 &\myIntX  \Omega { \nabla v \cdot \cof(D_h^2 u) \nabla w } \nonumber\\
 & -\sum\limits_{e \in \edges}\myIntS e { \jump {v \average { \cof(D_h^2 u) \nabla w} }}
 - \sum\limits_{e \in \edges}\myIntS e { \jump {w \average{ \cof(D^2 u) \nabla v} }} \nonumber\\  
 & +\sum\limits_{e \in \bigEps} \myIntS e { \frac \sigma {|e|} \jump {v}  \jump {w}}. \label{eq: linearisation our}
\end{align}
Comparing now \eqref{eq: linearisation our} with \eqref{eq: linearisation brenner} we notice that they are equal except for the interior edge integrals. But this difference is only due to the different choice of ansatz spaces: While in \cite{BGN+2011} Brenner et al. use $C^0$ elements, the Picard iteration is based on $DG$ elements. For the continuous ansatz functions all function jumps across interior edges vanish and hence all interior edge integrals equal to zero.

Consider $u\in H^s(\Omega)$ for $s>3$ to be the convex solution of the \MA problem. Brenner et al. formulated a nonlinear finite element method with the goal that its linearisation at the solution $u$ is given by \eqref{eq: linearisation brenner}. In other words their penalty method can be formulated by a mapping $F^B:H^3(\Omega; \triang) \rightarrow [H^2(\Omega; \triang) \cap H^1(\Omega)]'$ satisfying
 \begin{align}
 	F^B(u +w ) = L^B_u w + R^Bw \label{eq: add brenner method}
 \end{align}
 with a (nonlinear) mapping $R^B:H^3(\Omega; \triang) \rightarrow [H^2(\Omega; \triang) \cap H^1(\Omega)]'$. 
% They compute $R^B$ to be given by
% \begin{align}
% 	\bilin {R^B w} v 
% 	= & - \myInt_\Omega \detHessH{w} v dx 
% 		+ \sum_{e \in \edgesi} \jump {\average{\cofHessH{w}} \nabla w} v ds \nonumber \\
% 		&- \sum_{e\in \edgesb} \myIntS e { \cofHessH{w} \nabla v \cdot \mathbf{n} \; w ds 
% \end{align}.
% 
We want to derive a similar identity for the Picard iteration. 

For the rest of the analysis, we consider a convex and polyhedral domain $\Omega\subset \R^2$ and $f,g$ such that the \MA problem has a unique convex solution $u \in H^s(\Omega)$ with $s\geq 2$. 

Note that from now on $u$ denotes a fixed function. While Brenner et al. only linearise $F$ at the point $u$ the Picard method linearises at the last step's solution. Thus, $L_u^B$ defined in \eqref{eq: linearisation brenner} has always the same parameter $u$, namely the exact solution, whereas the parameter $u$ in the definition of $L_u$ (cf. \eqref{eq: linearisation our}) varies. 

To derive results on the Picard iteration, we follow the argumentative structure Brenner et al. used to show convergence of their method in \cite{BGN+2011}. At first, we introduce the functional $R_u \in V'$ for the right-hand side of the SIPG method defined by
\begin{align}
\bilin {-R_u} v = - 2 \myIntX  \Omega {v f}
-\sum\limits_{e \in \edgesb}\myIntS e { g \cofHessH{u} \nabla v \cdot \mathbf n }
+\sum\limits_{e \in \edgesb} \myIntS e { \frac \sigma {|e|} v g}. \label{eq: definition R}
\end{align}
The desired discretised Picard iteration then states: Find $u^{i+1}_h \in V_h$ such that
\begin{align}
\bilin {L_{u^i_h} w_h} {v_h} =  \bilin {-R_{u^i_h}} {v_h}   \qquad \forall v_h \in V_h. \label{eq: fixed point functional}
\end{align}
Formulating this as an operator equation yields: Find $u_h^{i+1} \in V_h$ being a root of the function $F:V \rightarrow V'$ defined by
\begin{align}
 {Fw} =  L_{u^i_h} w + R_{u^i_h}  \qquad \text{ for } w \in V. \label{eq: definition F}
\end{align}



Thus we have the identity 
\begin{align}
	F(u+w) = L_{u^i_h} (u+w) + R_{u^i_h} = L_{u^i_h} u +L_{u^i_h}w + R_{u^i_h} = Fu + L_{u^i_h}w. \label{eq: add our method}
\end{align}

Let us compare \eqref{eq: add brenner method} with \eqref{eq: add our method}: We have already seen the commonality of both methods, they base on the same linearisation and thereby gain the permutability between linearisation and discretisation. 
But there are also two major differences: While the right-hand side of \eqref{eq: add brenner method} contains a nonlinear mapping $R^B$, \eqref{eq: add our method} contains the additional term $Fu$. \\
Note that \eqref{eq: add brenner method} also be written as 
 \begin{align*}
 F^B(u +w ) = F^B u + L^B_u w + R^Bw,
 \end{align*}
since the exact \MA solution is a root of the finite element method operator. 
Hence, the distinction between both approaches can be characterised: While the first one reproduces the nonlinearity of the \MA equation with a nonlinear term, the second ansatz replaces parts of the nonlinear terms of the equation by an approximation and thus, trades nonlinearity for consistency errors.

\subsection{Consistency Estimates}

To measure the error in the second derivative we introduce a new quantity. Let $\epsilonTwoZeroArg:V \rightarrow \R$ be defined by
\[
	\epsilonTwo{v} := \max{ \LTwonorm{\hess v}}{ \sup_{e \in \edges} \LTwonormE{\average{\hess v}}}.
\]	

The following lemma shows $\epsilonTwo{\cdot}$ is not only a measure for the Hessian, but also for the cofactor of the Hessian.
\begin{lemma} \label{la: epsilon cof hess equal}
	For any convex $v \in V$ it holds
	\[
		\epsilonTwo{v} = \max{ \LTwonorm{\cofHessH v}}{ \sup_{e \in \edges} \LTwonormE{\average{\cofHessH{v}}}}.
	\]
\end{lemma}
\begin{proof}
	We have for a positive definite matrix $A$ that 
	\begin{align*}
	\LTwonorm{A} =  \sup_{v \in L^2(\Omega), \LTwonorm{v}=1} \myIntX \Omega {v^t A^t A v} = \lambda^2,
	\end{align*}
	where $\lambda$ is the largest eigenvalue of $A$. Since for every $w \in V$ the eigenvalues of $\cofHessH{w}$ and $\hess{w}$ coincide on every $T \in \triang$ we obtain 
	\begin{align}
	\LTwonormT{\cofHessH{w}} = \LTwonormT{\hess{w}} \qquad \forall T \in \triang.\label{eq: equality norm}
	\end{align}
	Furthermore using the definition of the average (cf. Definition \ref{def: edge operators}) and that $\cofHessH{w}$ and $\hess{w}$ are linear in $w$ for all $w \in V$ we find
	\begin{align}
	\LTwonormE{\average {\cofHessH{w}}} 
	=& \LTwonormE{\frac 1 2 {\cofHessH{(w^+ + w^-)}}} \nonumber\\
	=& \LTwonormE{\frac 1 2 {\hess{(w^+ + w^-)}}} 
	= \LTwonormE{\average {\hess{w}}} \qquad \forall e \in \edges. \label{eq: equality norm average}
	\end{align}
	\phantom{blub}
\end{proof}

Let for $w \in V$ the mapping $L_{w,h}:V_h \rightarrow V_h'$ be the restriction from $L_w$ to $V_h$. All further analysis is based on the following lemma.
\begin{lemma}[Stability] \label{la: stability L}
	Suppose $w \in V$ to be convex. Then 
	\begin{align}
		\HMinusOneDnorm{L_{w}v} \leq \epsilonTwo{w} \eNorm{v} \qquad \forall v \in V.
	\end{align}
	Further, for a penalty parameter $\sigma $ sufficiently large $L_{w,h}: V_h \rightarrow V_h'$ is invertible and satisfies
	\begin{align}
		\eNorm{L_{w,h}^{-1} r_h} \leq C \HMinusOneDnorm{r_h} \qquad \forall r_h \in V_h'.
	\end{align}
	$C$ denotes a positive constant independent on $h$ and $r_h$. 
\end{lemma}
\begin{proof}
	Since for convex $w$ the cofactor matrix is always positive definite the claim follows from the proof of Theorem \ref{thm: SIPG stability}.
\end{proof}

Note that this means $F$ defined in \eqref{eq: definition F} has a unique solution in $V_h$ as $R$ is constant in $w$.

For further analysis we want to examine the dependence of $L_w$ on the parameter $w$.
\begin{lemma}\label{la: L dependence parameter}
	For $w_1, w_2 \in V$ holds
	\[
		\HMinusOneDnorm{L_{w_1} v - L_{w_2} v} \leq \epsilonTwo{w_1 - w_2} \eNorm{v} \qquad \forall v \in V.
	\]
\end{lemma}
\begin{proof}
	Let $w_1, w_2, v$ be in $V$. For any $\varphi_h \in V_h$ we have by the definition of $L$ in \eqref{eq: linearisation our} and by linearity
	\begin{align}
		\bilin{L_{w_1}v- L_{w_2} v} {\varphi_h} =  
			&\myIntX  \Omega { \nabla \varphi_h \cdot \cofHessH{(w_1-w_2)} \nabla v}  \nonumber\\
			& -\sum\limits_{e \in \edges}\myIntS e { \jump {\varphi_h \average { \cofHessH{(w_1-w_2)} \nabla v} }} \nonumber \\
			& - \sum\limits_{e \in \edges}\myIntS e { \jump {v \average{ \cofHessH{(w_1-w_2)} \nabla \varphi_h} }}. \label{eq: diff linear}
	\end{align}
	As already done in the proof of Theorem \ref{thm: SIPG stability} we use the Cauchy-Schwarz inequality and the discrete Cauchy-Schwarz inequality on every summand and obtain that $\bilin{L_{w_1}v- L_{w_2} v} {\varphi_h}$ is smaller than
	\begin{align*}
%		\eqref{eq: diff linear}%\bilin{L_{w_1}v- L_{w_2} v} {\varphi_h} 
%		\leq
		& \LTwonorm{\nabla \varphi_h} \LTwonorm{\cofHessH{(w_1 -w_2)}} \LTwonorm {\nabla v} \nonumber \\
			&+ \left( \sum_{e \in \edges} \frac {\sigma}{|e|}\LTwonormE{\jump {\varphi_h}}^2 \right)^{\frac 1 2}
			   \left( \sum_{e \in \edges} \frac {|e|} \sigma\LTwonormE{\average{\cofHessH{(w_1-w_2} \nabla v}}^2 \right)^{\frac 1 2} \\
			&+ \left( \sum_{e \in \edges} \frac {\sigma}{|e|}\LTwonormE{\jump v}^2 \right)^{\frac 1 2}
			\left( \sum_{e \in \edges} \frac {|e|} \sigma\LTwonormE{\average{\cofHessH{(w_1-w_2} \nabla \varphi_h}}^2 \right)^{\frac 1 2}
	\end{align*}
	and hence with Lemma \ref{la: epsilon cof hess equal}
	\begin{align*}
		\bilin{L_{w_1}v- L_{w_2} v} {\varphi_h}
		\leq& \epsilonTwo{w_1 - w_2} \\
			&\times 
			\left( \LTwonorm{\nabla \varphi_h} \LTwonorm {\nabla v} 
				\phantom{\left( \sum_{e \in \edges} \frac {|e|} \sigma\LTwonormE{\average{\nabla \varphi_h}}^2 \right)^{\frac 1 2}}
			 \right.\nonumber \\
			&\phantom{\times \;\;\;}+ \left( \sum_{e \in \edges} \frac {\sigma}{|e|}\LTwonormE{\jump {\varphi_h}}^2 \right)^{\frac 1 2}
			\left( \sum_{e \in \edges} \frac {|e|} \sigma\LTwonormE{\average{\nabla v}}^2 \right)^{\frac 1 2} \\
			&\phantom {\times \;\;\;}+ \left( \sum_{e \in \edges} \frac {\sigma}{|e|}\LTwonormE{\jump v}^2 \right)^{\frac 1 2}
			\left. \left( \sum_{e \in \edges} \frac {|e|} \sigma\LTwonormE{\average{\nabla \varphi_h}}^2 \right)^{\frac 1 2} \right).
	\end{align*}
	Using now the discrete Cauchy-Schwarz inequality on the whole term yields
	\begin{align*}
		\bilin{L_{w_1}v- L_{w_2} v} {\varphi_h} 
		\leq& \epsilonTwo{w_1 - w_2} \\
		  & \times 
			\left( 
				\LTwonorm{\nabla \varphi}^2
					+ \sum\limits_{e \in \edges} \frac {\sigma} {|e|}\LTwonormE{\jump {\varphi}}^2
					+ \sum\limits_{e \in \edges} \frac {|e|} \sigma \LTwonormE{\average{\nabla \varphi}}^2
				\right)^{\frac 1 2} \nonumber \\
			&\times
			\left( 
				\LTwonorm{\nabla v}^2
					+\sum\limits_{e \in \edges} \frac {\sigma}{|e|}\LTwonormE{\jump {v}}^2
					+ \sum\limits_{e \in \edges} \frac {|e|} \sigma \LTwonormE{\average{\nabla v}}^2
			\right) ^{\frac 1 2} \nonumber \\
			\leq & \epsilonTwo{w_1 - w_2} \eNorm \varphi \eNorm v.
	\end{align*}
	The result follows then with Definition \ref{def: h-1 seminorm}.
%	Since the eigenvalues of $\cofHessH{(w_1-w_2}$ and $\hess{(w_1-w_2)}$ coincide their $L^2$ norm also coincides. Hence, by this and the discrete Cauchy-Schwarz inequality it follows
%	\begin{align}
%		\eqref{eq: diff linear 2} %\bilin{L_{w_1}v- L_{w_2} v} {\varphi_h} 
%		\leq& \LTwonorm{\nabla \varphi_h} \LTwonorm{\hess{(w_1 -w_2)}} \LTwonorm {\nabla v} \nonumber \\
%			&+ \LInftynorm{\jump {\varphi_h}} 
%				\left(\sum_{e \in \edges} \LTwonormE{\average{\hess{(w_1-w_2}}}^2 \right)^{\frac 1 2} 
%				\left(\sum_{e \in \edges} \LTwonormE{\average{\nabla v}}^2 \right)^{\frac 1 2} \nonumber \\
%			&+ 	\left(\sum_{e \in \edges} \LTwonormE{\jump v}^2 \right)^{\frac 1 2} 
%				\left(\sum_{e \in \edges} \LTwonormE{\average{\hess{(w_1-w_2}}}^2 \right)^{\frac 1 2} 
%				\LInftynorm{\average{\nabla \varphi_h}}. \label{eq: estimate diff L}
%	\end{align}
%	By the definition of jump and average it is clear that 
%	\[
%		\LInftynorm{\jump {\varphi_h}} = \LInftynorm{\varphi_h}\text{ and }\LInftynorm{\average{\nabla \varphi_h}}= \LInftynorm{\nabla \varphi_h}
%	\]
%	hold. Further we can estimate the $L^\infty$ norm using the discrete Sobolev inequality (\todo{discrete sobolev}) and we find
%	\begin{align}
%		\LInftynorm{\average {\nabla \varphi_h}} \leq C (1+|\ln h|^{\frac 1 2}) \LTwonorm{\nabla \varphi_h} \leq C (1+|\ln h|^{\frac 1 2}) \eNorm{\nabla \varphi_h}, \label{eq: estimate average phi}
%	\end{align}
%	and
%	\begin{align}
%		\LInftynorm{\jump {\varphi_h}} \leq C (1+|\ln h|^{\frac 1 2}) \LTwonorm{\varphi_h}. \label{eq: first estimate jump phi}
%	\end{align}
%	To further estimate \eqref{eq: first estimate jump phi} we apply the inverse estimate from Lemma \ref{la: inverse estimate} and the Poincar\'e inequality (\todo{poincare inequality}) and obtain
%	\begin{align}
%		\LTwonorm{\varphi_h} = \sum_{T \in \triang} \LTwonormT{\varphi_h} \leq C h \sum_{T \in \triang} \HOnenormT{\varphi_h} \leq C h \sum_{T \in \triang} \singleNorm{\varphi_h}_{H^1(T)} \leq Ch \eNorm{\varphi_h} \label{eq: estimate phi}
%	\end{align}
%	By \eqref{eq: estimate average phi}, \eqref{eq: first estimate jump phi} and \eqref{eq: estimate phi} and $h<1$, we have
%	\begin{align}
%		\maxTriple{\LTwonorm{\nabla \varphi}}{\LInftynorm{\jump{\varphi_h}}}{\LInftynorm{\average{\nabla \varphi_h}}} \leq C (1+|\ln h|^{\frac 1 2}) \eNorm{\varphi_h}
%	\end{align} 
%	and hence by \eqref{eq: estimate diff L}
%	\begin{align}
%		\bilin{L_{w_1}v- L_{w_2} v} {\varphi_h} 
%		\leq& C (1+|\ln h|^{\frac 1 2})\eNorm{\varphi} 
%			\left[ 
%				\LTwonorm{\hess{(w_1 -w_2)}} \LTwonorm {\nabla v} \nonumber \right.\\
%				&+ 	\left(\sum_{e \in \edges} \LTwonormE{\average{\hess{(w_1-w_2}}}^2 \right)^{\frac 1 2} 
%					\left(\sum_{e \in \edges} \LTwonormE{\average{\nabla v}}^2 \right)^{\frac 1 2} \nonumber \\
%				&+ 	\left(\sum_{e \in \edges} \LTwonormE{\jump v}^2 \right)^{\frac 1 2} 
%				\left. \left(\sum_{e \in \edges} \LTwonormE{\average{\hess{(w_1-w_2}}}^2 \right)^{\frac 1 2} 
%			\right] \label{eq: difference estimate number two}
%	\end{align}
%\todo{brackets}	
%	Using the discrete Cauchy-Schwarz inequality we find
%	\begin{align}
%	\bilin{L_{w_1}v- L_{w_2} v} {\varphi_h} 
%		\leq& C (1+|\ln h|^{\frac 1 2})\eNorm{\varphi} 
%		\nonumber \\
%		&\times
%			\left(
%				\LTwonorm{\hess{(w_1 -w_2)}}^2 
%				+ 2\sum_{e \in \edges} \LTwonormE{\average{\hess{(w_1-w_2}}}^2
%			\right)^{\frac 1 2} \nonumber \\
%	   	&\times
%			\left(
%			\LTwonorm{\nabla v}^2 
%			+ \sum_{e \in \edges} \frac {|e|}\sigma \LTwonormE{\average{\nabla v}}^2
%			+\sum_{e \in \edges} \frac \sigma {|e|}\LTwonormE{\jump {v}}^2
%			\right)^{\frac 1 2} \nonumber \\
%		\leq& C (1+|\ln h|^{\frac 1 2})\eNorm{\varphi} 	\epsilonTwo{{(w_1-w_2)}} \eNorm{v}
%	\end{align}
% \phantom{blub}
\end{proof}


Now we are able to establish an estimate of the consistency error of $Fu$.
\begin{theorem}[Consistency Error of $Fu$] \label{la: consistency error F}
	There holds
	\begin{align*}
	\HMinusOneDnorm{Fu} \leq& C (\eNorm{u}+1)\epsilonTwo{u-u_h^i},
	\end{align*}
	where $C$ is a positive constant that depends on the mesh, $g$ and $u^i_h$, but not on $h$ and $u$. 
\end{theorem}
\begin{proof}
	We have for $v \in V_h$
	\begin{align}
	%		\bilin{Fu} v = \myIntX  \Omega {
	\HMinusOneDnorm{Fu} =& \HMinusOneDnorm{ L_{u^i_h}u + R_{u^i_h}} \nonumber\\
	\leq& \HMinusOneDnorm{ L_{u^i_h}u - L_u u} + \HMinusOneDnorm{ L_{u}u + R_u} + \HMinusOneDnorm{ -R_u + R_{u^i_h}}. \label{eq: Fu first estimate}
	\end{align}	
	The exact solution $u$ is a fixed point of $- \nabla \cdot \left( \mycof{ D^2 u^i} \nabla u^{i+1} \right)  = -2f$ (cf. \eqref{eq:fixed point iteration}). The mappings $L_w$ and $R_w$ were defined to formulate the SIPG method for this equation (cf. \eqref{eq: fixed point functional}). Thus, we obtain
	\begin{align}
	L_{u} u + R_u = 0. \label{eq: right solution L+U}
	\end{align}
	By the definition of $R$ in \eqref{eq: definition R} and the Cauchy-Schwarz inequality we find for any $v \in V_h$ 
	\begin{align}
	\bilin {R_u - R_{u^i_h}} v 
	=& \sum_{e \in \edgesb} \myIntX \Omega {g \cofHessH{(u-u_h^i)} \nabla v \cdot \mathbf n }\nonumber \\
	\leq& \LInftynormPOm{g}
		\sum_{e \in \edgesb} \LTwonormE{ \cofHessH{(u-u_h^i)} \nabla v }^2 \nonumber \\
	\leq& \LInftynormPOm{g}
		\left( \sum_{e \in \edgesb} \frac {|e|} \sigma \LTwonormE{ \cofHessH{(u-u_h^i)}}^2  \right)^{\frac 1 2}
		\left( \sum_{e \in \edgesb} \frac \sigma {|e|} \LTwonormE{\nabla v }^2  \right)^{\frac 1 2}. \label{eq: R first estimate}
	\end{align}
	Using again \eqref{eq: equality norm}, namely that the $L^2$ norm of a matrix and its cofactor matrix coincide, as well as the standard combination of the trace estimate from Lemma \ref{la: trace estimate} and the inverse estimate from Lemma \ref{la: inverse estimate}\footnote{for more detailed steps of the combination of trace and inverse estimate see the proof of Lemma \ref{la: equivalence energy norm}} we find
	\begin{align}
	\sum_{e \in \edgesb} \frac 1 {|e|} \LTwonormE{ \cofHessH{(u-u_h^i)}}^2 
	\leq C \sum_{T \in \triang} \LTwonormT{\hess{(u-u_h^i)}}^2.
	\end{align}
	Combining this with \eqref{eq: R first estimate} yields
	\begin{align*}
	\bilin {R_u - R_{u^i_h}} v \leq C \eNorm{v} \epsilonTwo{u-u_h^i}
	\end{align*}
	and hence 
	\begin{align}
	\HMinusOneDnorm{R_u - R_{u^i_h}} \leq C \epsilonTwo{u-u_h^i}. \label{eq: contraction R}
	\end{align}
	The consistency error estimate then follows from \eqref{eq: Fu first estimate}, Lemma \ref{la: L dependence parameter}, \eqref{eq: right solution L+U} and \eqref{eq: contraction R}.
\end{proof}

This result is very important to understand the derived method. It states that the consistency error depends linearly on the error $\epsilonTwo{u-u^i_h}$. That means the method is only consistent if $\epsilonTwo{u-u^i_h}$ depends linearly on $h^l$ for a $l \geq 1$. Otherwise the consistency error is not controllable and in particular the introduced method is not consistent.

\subsection{Error estimates}
Now we want to transfer the argument structure given by Brenner et al. to derive error estimates. So, analogously to their definition in \cite[(3.1)]{BGN+2011} we define the operator $\mathcal M: V \rightarrow V_h$ by
\begin{align}
	\mathcal M = L_{u,h}^{-1}(L_{u} - F).
\end{align}
Let $\mathcal M_h:V_h \rightarrow V_h$ be its restriction to $V_h$. Clearly, $\mathcal M_h$ reduces to 
\begin{align}
\mathcal M = id_h - L_{u,h}^{-1}F.
\end{align}
Since $L_{u,h}$ is a isomorphism every fixed point of $\mathcal M_h$ is also a root of $F$ and hence a fixed point to the Picard iteration.

We want to derive a contraction property as well as a mapping property to apply Banach's fixed point theorem. If we are able to show that $\M_h$ restricted to an arbitrary small set is a self-mapping we can locate the fixed point with arbitrary precision. Like Brenner et al. \cite[(3.3)]{BGN+2011} we choose this set to be a small ball centred around $u_{c,h}\in V_h$, where
\[
u_{c,h} = L_{u,h}^{-1} L_u u.
\]

By \eqref{eq: add our method} for all $w \in V$
\begin{align}
	\mathcal M w &= L_{u,h}^{-1}(L_u w - Fw) \nonumber\\
				 &= L_{u,h}^{-1}(L_u w - F(u+w-u)) \nonumber\\
				 &= L_{u,h}^{-1}(L_u w - Fu - L_{u^i_h} (w-u)) \nonumber\\
				 &=  L_{u,h}^{-1}(L_u w - L_{u^i_h} w + L_{u^i_h}u - L_u u + L_u u - Fu) \nonumber\\
				 & = u_{c,h} + L_{u,h}^{-1}(L_u w - L_{u^i_h} w + L_{u^i_h}u - L_u u - Fu) \label{eq: expand M}
 \end{align}
and hence for $w_1, w_2 \in V$
\begin{align*}
	\eNorm{\mathcal M w_1 - \mathcal M w_2} =& \eNorm{L_{u,h}^{-1}(L_u (w_1-w_2) - L_{u^i_h} (w_1-w_2))}.
\end{align*}
Applying Lemma \ref{la: stability L} and Lemma \ref{la: L dependence parameter} yields the following contraction result which is motivated by \cite[Lemma 3.4]{BGN+2011}.
\begin{lemma}[Contraction property of $\mathcal{M}$] \label{la: contraction property M}
	For any $w_1, w_2 \in V$ it holds for a positive constant $C$ independent on $h$
\begin{align*}
	\HOneDnorm{\mathcal M w_1 - \mathcal M w_2}	\leq C \epsilonTwo{u - u^i_h} \eNorm{w_1 - w_2}. \label{eq: estimate M}
\end{align*}
\end{lemma}
It is easy to see that the contraction property of $\mathcal{M}$ depends strongly on the approximation quality of the last step's solution. %If this estimation is sharp it also means that 

Before we prove the self-mapping property, we want justify the choice of $u_{c,h}$. Therefore we show $u_{c,h}$ is close to the exact \MA solution.

\begin{lemma}\label{la: difference u uch}
	The distance between the exact solution $u$ and $u_{c,h}$ is mainly determined by the approximation properties of $V_h$, namely
	\[
		\eNorm{u-u_{c,h}} \leq C \inf_{v_h \in V_h} \eNorm{u-v_h},
	\]
	where $C>1$ denotes a constant independent on $h$, but dependent on $u$.
\end{lemma}
\begin{proof}
	For any $v_h \in V_h$ holds applying the definition of $u_{c,h}$ and both statements given in Lemma \ref{la: stability L}
	\begin{align}
		\eNorm{u-u_{c,h}} \leq& \eNorm{u- v_h} + \eNorm{ v_h- L_{u,h}^{-1}L_u u} \nonumber \\
			=& \eNorm{u- v_h} + \eNorm{L_{u,h}^{-1}L_{u,h} v_h- L_{u,h}^{-1}L_u u} \nonumber \\
			\leq& \eNorm{u- v_h} + C \HMinusOneDnorm{L_{u,h} v_h-  L_u u} \nonumber \\
			\leq& \eNorm{u- v_h} + C \epsilonTwo{u}\eNorm{v_h -u}. \nonumber
	\end{align}
	Since $v_h \in V_h$ was arbitrary the desired estimate follows.
\end{proof}

Let us now derive an estimate for the image of a small ball centred at $u_{c,h}$ inspired by the estimate \cite[Lemma 3.6]{BGN+2011}.
\begin{lemma}[Mapping Property of $\M$] \label{la: mapping property of M}
	Let 
	\[
		\mathbb B_\rho(u_{c,h}):=\{v_h \in V_h: \eNorm{v_h - u_{c,h}} \leq \rho\}
	\]
	denote a small ball in $V_h$ centred at $u_{c,h}$. 
	Then, there holds for any $v \in \mathbb B_\rho(u_{c,h})$
	\[
		\eNorm{u_{c,h} - \M_h v} \leq C \epsilonTwo{u - u^i_h} (\rho+ \inf_{v_h \in V_h} \eNorm{u-v_h}+1),
	\]
	where $C$ is a positive constant independent on $h$, but dependent on $u$.
\end{lemma}
\begin{proof}
By \eqref{eq: expand M} we have for any $w \in V$
	\begin{align}
		\eNorm{u_{c,h} - \M_h w} = \eNorm{L_{u,h}^{-1}(L_u w - L_{u^i_h} w + L_{u^i_h}u - L_u u - Fu)}
	\end{align}
	and hence using Lemma \ref{la: stability L} and \ref{la: L dependence parameter}
	\begin{align*}
		\eNorm{u_{c,h} - \M_h w} 
		\leq& C\HMinusOneDnorm{(L_u w - L_{u^i_h} w + L_{u^i_h} u- L_u u - Fu)} \nonumber\\
		\leq& C\HMinusOneDnorm{(L_u (w-u) - L_{u^i_h} (w-u) - Fu)} \nonumber\\
				\leq& C \epsilonTwo{u - u^i_h} \eNorm{w-u} +\HMinusOneDnorm{Fu}.
	\end{align*}
	Using the result in Lemma \ref{la: consistency error F} we obtain
	\begin{align*}
		\eNorm{u_{c,h} - \M_h w} \leq& C \epsilonTwo{u - u^i_h} (\eNorm{w-u}+\eNorm{u} +1)\nonumber \\
		\leq & C \epsilonTwo{u - u^i_h} (\eNorm{w-u}+1).
	\end{align*}
	Consider now $v_h \in \mathbb B_\rho(u_{c,h})$, by the triangle inequality we find 
	\begin{align}
		\eNorm{u_{c,h} - \M_h v} 
			\leq& C \epsilonTwo{u - u^i_h} (\eNorm{v_h-u_{c,h}}+ \eNorm{u_{c,h}-u} +1) \nonumber \\
			\leq& C \epsilonTwo{u - u^i_h} (\rho+ \eNorm{u_{c,h}-u} +1).
	\end{align}
Using the estimate of Lemma \ref{la: difference u uch} implies the claimed estimate.
\end{proof}

This result is inspired by Lemma 3.6 in \cite{BGN+2011}. Once again, we observe the differences of the two methods: While the statement in this Lemma relies on the previous consistency estimates, the estimate derived by Brenner et al. depends on the contraction property of the nonlinear part $R^B$ from \eqref{eq: add brenner method}. 

%which states 
%\begin{align}
%	\eNorm{u^B_{c,h} - \mathcal{M}^B_h v} \leq C_1 h^{-2} (1+|\ln h|^{\frac 1 2} ) \left(\rho^2+C_2 h^{2(l-1)} \norm u_{H^l(\Omega)}^2 \right) \label{eq: estimate Brenner}
%\end{align}
%Note that $u^B_{c,h}$ and $\mathcal{M}^B_h v$ are analogously to $u_{c,h}$ and $\mathcal{M}_h$ defined using the $L^B$ and $R^B$ from \ref{eq: add brenner method}. We observe that 
%%

Now we are able to combine the preceding results and thereby derive an estimate as in the main theorem of \cite{BGN+2011}.
\begin{theorem}\label{main result}
	Given a sufficiently large $\sigma$ and $u_h^i$ sufficiently close to the exact solution, there exists a solution $u_h$ to the penalty method \eqref{eq: fixed point functional} satisfying
	\begin{align}
		\eNorm{u-u_h} \leq C h^{s-1}, \label{eq: error estimate}
	\end{align}
	where $0 \leq s \leq \min\{k+1, t\}$ for $u$ contained in $H^t(\Omega)$. Hereby $C$ denotes a positive constant depending on the mesh, $g$, $\sigma$, $u^i_h$, and $u$, but not on $h$.\\
	Furthermore if there exists $C^*$ independent on $h$ such that
	\begin{align}
		\epsilonTwo{u-u_h^i} \leq C^* h^{s}, \label{eq: estimate error}
	\end{align}
	then there exist a $h^*> 0$ such that $u_h^i$ fulfils the requirements to apply \eqref{eq: error estimate} for all $h > h^*$.
	
\end{theorem}
\begin{proof}
	Fix $h$ and take
	\begin{align}
		\rho_0:= h^{s-1} \norm{u}_{H^{s}(\Omega)}. \label{eq: definition rho0}
	\end{align}
	
	Define 
	\begin{align}
		\tau_1 &= 2 C_1 \epsilonTwo{u-u_h^i}, \label{eq: tau1}\\
		\tau_2 & = 2 C_2 \epsilonTwo{u-u_h^i}, \label{eq: tau2}\\
		\tau_3 & = C_2 \epsilonTwo{u-u_h^i} \left(\inf_{v_h \in V_h} \eNorm{u-v_h}+1\right) 2 \rho_0^{-1} \label{eq: tau3},
	\end{align}
	where $C_1$ is the constant of Lemma \ref{la: contraction property M} and $C_2$ the constant of Lemma \ref{la: mapping property of M}. 
	
	Let $u_h^i$ fulfil
	\begin{align}
		%\max
		\delta := \maxTriple {\tau_1} {\tau_2} {\tau_3}< 1.
	\end{align}
	Thereby it follows from \eqref{eq: tau1} that $\M$ fulfils the contraction property stated in Lemma \ref{la: contraction property M} with the constant $\tau_1\leq \delta<1$.
	
	Lemma \ref{la: mapping property of M} implies for any $v_h \in \mathbb{B}_{\rho_0}(u_{c,h})$
	 \[
	 	\eNorm{u_{c,h} - \M_h v_h} \leq 
	 		C_2 \epsilonTwo{u - u^i_h} (\rho_0+ \inf_{v_h \in V_h} \eNorm{u-v_h}+1).
	 \]
	By \eqref{eq: tau2} we have 
	 \[
	 	\eNorm{u_{c,h} - \M_h v_h} \leq 
	 		\frac \delta 2 \rho_0 
	 			+ C_2  \epsilonTwo{u - u^i_h} \left( \inf_{v_h \in V_h} \eNorm{u-v_h}+1\right)
	 \]
	 and then by \eqref{eq: tau3} we obtain
	 \begin{align}
	 	\eNorm{u_{c,h} - \M_h v_h} \leq 
	 		\frac \delta 2 \rho_0 
	 		+ \frac {\rho_0} 2  < \rho_0. \label{eq: self mapping M}
	 \end{align} 
	 
	 Now we fulfil all requirements to apply Banach's fixed point theorem: $\mathbb{B}_{\rho_0}(u_{c,h}), $ is a closed subset of the Banach space $(V_h, \eNorm{\cdot})$, and $\mathcal M_h$ is a contraction mapping on $ \mathbb{B}_{\rho_0}(u_{c,h})$. Hence, the theorem yields the existence of a unique fixed point $u_h$ of $\M_h$ in $\mathbb B_{\rho_0}$.
	 Moreover, by Lemma \ref{la: difference u uch} it holds
	 \begin{align}
	 	\eNorm{u-u_h} \leq \eNorm{u-u_{c,h}} + \eNorm{u_{c,h} - u_h} \leq C \inf_{v_h \in V_h} \eNorm{u-v_h} +\rho_0.
	\end{align}
	\eqref{eq: error estimate} then follows by the approximation property given in Lemma \ref{la: approximation properties energy norm}.
	
	Given $\epsilonTwo{u-u_h^i} \leq h^{s+1} \norm{u}_H^s(\Omega)$ we find using the definitions of $\rho_0$ and $\tau_3$ (cf. \eqref{eq: definition rho0} and \eqref{eq: tau3})
	\begin{align*}
		\tau_3  \leq C h^{s} \left(h^{s-1} \norm{u}_{H^s(\Omega)}+1\right) h^{-s+1} \norm{u}_{H^{s}(\Omega)}^{-1} \xrightarrow{h \rightarrow 0}  0.
	\end{align*}
	Since analogously $\tau_1, \tau_2 \xrightarrow[]{h\rightarrow 0} 0$ holds there exist $h^*$ such that their maximum, namely $\delta$, is smaller than 1.
\end{proof}
	
Note that $\mathcal P^k_h$ for $u \in H^t(\Omega)$ has the approximation property  (cf. Lemma \ref{la: approximation properties})
\begin{align}
	\inf_{v \in V_h} \sum_{T \in \triang} \singleNorm{u-v_h}_{H^2(T)} \leq h^{l-2} \singleNorm{u}_{H^l(\Omega)}, \label{eq: approx hess}
\end{align}
where $l = \min\{k+1,t\}$. $\singleNorm{u-u^i_h}_{H^2(\Omega)} = \LTwonorm{\hess {(u-u^i_h)}}$ influences $\epsilonTwo{u-u^i_h}$ and thus, Theorem \ref{main result} yields even for a reliable approximation of $u$ no satisfying result on the Picard Iteration. Furthermore, if we have a guess that fulfils \eqref{eq: estimate error} this theorem does not ensure $u_h$ is a better approximation to $u$ than $u^i_h$. 

The analogue result for the method of Brenner et al. \cite[Theorem 3.1]{BGN+2011} yields optimal convergence results, but they depend on the property that $l$ defined as in \eqref{eq: approx hess} is strictly greater than $3$ to control the mapping property of $\mathcal M$. However, we depend on the error in the second derivatives, i.e. $\epsilonTwo{u-u_h^i}$ to control this mapping property. The problem is that derived mapping property in Lemma \ref{la: mapping property of M} is not sufficient to imitate the results of Brenner et al. as it does not decrease linearly with a power of $h$.

Unfortunately, later numerical results suggest that this formulation of a Picard Iteration do not converge to the \MA solution.	
	
