%\section{Well-posedness of the SIPG Method}
%
%In this section we want to prove th derived SIPG method is well-posed. The norm to carry out an analysis easily is the discrete energy norm.
%Let $V$ be $H^1(\Omega; \triang)$ and the bilinear form $\bilin \cdot \cdot_A:V \times V$ be given by
%\begin{align}
%	\bilin \varphi v _ A =  \sum_{T \in \triang} \myIntX  T { \nabla \varphi \cdot A \nabla v
%\end{align}
%\begin{definition}[Energy Norm] \label{def: energy norm}
%		The mesh-dependent energy norm for a function $v \in V$ is defined by
%		\[
%			\eNorm v ^2 := \bilinA v v + \frac 1 \sigma \sum_{e \in \edges} |e|\LTwonormE{\average{A \nabla v}} + 2  \sigma \sum_{e \in \edges} \frac 1 {|e|}\LTwonormE{\jump{\nabla v}}.
%		\]
%\end{definition}
%\todo{$\Omega,\triang$ erklaren}
%
%Note that through this section $C$ denotes a generic positive constant independent of $h$ that may take different values in each equation.
%
%At first, we show the following statement
%\begin{lemma}[Boundedness of the SIPG method]\label{la: SIPG continuous}
%	For the SIPG method holds
%	\[
%		|a_S(\varphi, v) + J^\sigma(\varphi,v)| \leq \eNorm{\varphi} \eNorm{v} \qquad \forall \varphi,v \in V_h
%	\]
%\end{lemma}
%\begin{proof}
%Let us consider the terms of \eqref{eq:inner product SIPG} separately.
%For the first term we have by the Cauchy-Schwarz inequality
%\begin{align}
%	|\sum\limits_{T \in \triang} \myIntX  T { \nabla \varphi \cdot A \nabla v| =& |\bilinA {\nabla \varphi} {\nabla v} | \leq \left( {\bilinA {\nabla \varphi} {\nabla \varphi}}\right)^{\frac 1 2 } \left( {\bilinA {\nabla v}{\nabla v}}\right)^{\frac 1 2 }. \label{eq: CS estimate 0}
%\end{align}
%Considering the sum of the second and the fourth term we find using Cauchy-Schwarz and using also the discrete Cauchy-Schwarz inequality
%\begin{align}
%	|\sum\limits_{e \in \edges}\myIntS e { \jump {\varphi \average{A \nabla v} }| &\leq
%	\sum\limits_{e \in \edges}  \LTwonormE{\jump {\varphi}} \LTwonormE {\average{A \nabla v}} \nonumber \\
%	& \leq
%		\left( \sum\limits_{e \in \edges} \frac {\sigma}{|e|} \LTwonormE{\jump {\varphi}}^2 \right)^{\frac 1 2}
%		\left( \sum\limits_{e \in \edges} \frac {|e|} \sigma \LTwonormE{\average{A \nabla v}}^2 \right)^{\frac 1 2} \label{eq: CS estimate}
%\end{align}
%Since the third and the last term in \eqref{eq:inner product SIPG} are the symmetrising terms, we analogously have for their sum
%\begin{align}
%\sum\limits_{e \in \edges}\myIntS e { \frac 1 {|e|}\jump {v \average{A \nabla \varphi} } \leq&
%			\left( \sum\limits_{e \in \edges} \frac {\sigma}{|e|}\LTwonormE{\jump {v}}^2 \right)^{\frac 1 2}
%			\left( \sum\limits_{e \in \edges} \frac {|e|} \sigma \LTwonormE{\average{A \nabla \varphi}}^2 \right)^{\frac 1 2}.\label{eq: CS estimate2}
%\end{align}
%Now only the last term of \ref{eq:inner product SIPG} needs to be estimated. Again we apply at first the Cauchy-Schwarz inequality and then the discrete Cauchy-Schwarz inequality:
%\begin{align}
%\singleNorm {\sum_{e \in \edgesi} \frac \sigma {|e|} \int_e \jump \varphi \jump v }
%\leq& \sum_{e \in \edgesi} \frac \sigma  {|e|} \int_e \singleNorm {\jump \varphi \jump v }
%\leq \left( \sum\limits_{e \in \edges} \frac \sigma{|e|} \LTwonormE{\jump{\varphi}}^2 \right)^{\frac 1 2} \left( \sum\limits_{e \in \edges} \frac \sigma {|e|} \LTwonormE{\jump{v}}^2 \right)^{\frac 1 2} \label{eq: CS estimate3}
%\end{align}
%Adding \eqref{eq: CS estimate 0}, \eqref{eq: CS estimate}, \eqref{eq: CS estimate2} and \eqref{eq: CS estimate3} we find using the discrete Schwarz inequality once more
%\begin{align}
%	|a_S(\varphi,v)| \leq& 
%		\left( {\bilinA {\nabla \varphi} {\nabla \varphi}}\right)^{\frac 1 2 } \left( {\bilinA {\nabla v}{\nabla v}}\right)^{\frac 1 2 } \nonumber\\
%	& +\left( \sum\limits_{e \in \edges}\frac {\sigma}{|e|}\LTwonormE{\jump {\varphi}}^2 \right)^{\frac 1 2}
%			\left( \sum\limits_{e \in \edges} \frac {|e|} \sigma \LTwonormE{\average{A \nabla v}}^2 \right)^{\frac 1 2} \nonumber\\
%	& +	\left( \sum\limits_{e \in \edges} \frac {\sigma} {|e|}\LTwonormE{\jump {v}}^2 \right)^{\frac 1 2}
%				\left( \sum\limits_{e \in \edges} \frac {|e|} \sigma \LTwonormE{\average{A \nabla \varphi}}^2 \right)^{\frac 1 2}\nonumber \\
%	& + \left( \sum\limits_{e \in \edges} \frac \sigma{|e|} \LTwonormE{\jump{\varphi}}^2 \right)^{\frac 1 2} \left( \sum\limits_{e \in \edges} \frac \sigma {|e|} \LTwonormE{\jump{v}}^2 \right)^{\frac 1 2} \nonumber \\
%	\leq& 
%	\left( 
%		\bilinA {\nabla \varphi} {\nabla \varphi}+2\sum\limits_{e \in \edges} \frac {\sigma} {|e|}\LTwonormE{\jump {\varphi}}^2+ \sum\limits_{e \in \edges} \frac {|e|} \sigma \LTwonormE{\average{A \nabla \varphi}}^2
%	\right)^{\frac 1 2} \nonumber \\
%	&\times
%		\left( 
%			\bilinA {\nabla v} {\nabla v}+2\sum\limits_{e \in \edges} \frac {\sigma}{|e|}\LTwonormE{\jump {v}}^2+ \sum\limits_{e \in \edges} \frac {|e|} \sigma \LTwonormE{\average{A \nabla v}}^2
%		\right) ^{\frac 1 2} \\
%		= & \eNorm \varphi \eNorm v 
%\end{align}
%\end{proof}
%
%The next step in our analysis considers the stability of the SIPG method. To do we first need to define a norm measuring ???\todo{yeah what} 
%\begin{definition}[$H^{-1}$ Semi-norm] \label{def: h-1 seminorm}
%	The semi-norm is given by 
%	\[
%		\HMinusOneDnorm r ^2 := \sup_{0 \neq w \in V_h} \frac {{\bilin r w } } {\eNorm{w}}.
%	\]
%\end{definition}
%Before we can start we need to define yet another mesh-dependent energy norm which later simplify the proof that the left-hand side's bilinear form of the SIPG method is coercive.
%\begin{definition}[Energy Norm]\label{def: energy norm2}
%The mesh-dependent discrete energy norm $\eNormTwo \cdot $ is defined by
%\[
%		\eNormTwo v^2 :=  \bilinA v v +  \sigma \sum_{e \in \edges} \frac 1 {|e|}\LTwonormE{\jump{\nabla v}}.
%\]
%\end{definition}
%Calling this norm also an energy norm is justified as both norms are equivalent on $V_h$.
%\begin{lemma}[Equivalence of Energy Norms in $V_h$]
%	For any $w \in V_h$ holds
%		\[
%			c \eNormTwo w \leq \eNorm w \leq C \eNormTwo w
%		\]
%		with positive Constants $c, C$ depending only on the mesh.
%\end{lemma}
%\begin{proof}
%	By definition the inequality $c \eNormTwo w \leq \eNorm w$ is clear and holds for example for $c=1$. 
%By the trace estimate we have
%	\begin{align}
%		\sum_{e \in \edges} \singleNorm e \LTwonormE{\average{A \nabla v}} 
%		\leq C \sum_{T \in \triang} \left( \LTwonormT{\average{A \nabla v}} +\HOnenormT{\average{A \nabla v}}   \right).
%	\end{align}
%Recalling the definition of the average (cf. Definition \ref{def: edge operators}) it is clear we can further simplify to
%		\begin{align}
%			\sum_{e \in \edges} \singleNorm e \LTwonormE{\average{A \nabla v}} 
%			\leq C \sum_{T \in \triang} \left( \LTwonormT{A \nabla v} +h^2 \HOnenormT{ A \nabla v}   \right).
%		\end{align}
%
%\end{proof}
%
%\begin{theorem}[Stability]\label{thm: SIPG stability}
%	There holds for $a_h v :=a_S(\cdot, v)+J^\sigma(\cdot, v)$
%	\begin{align*}
%	\HMinusOneDnorm{a_h v} \leq \eNorm v \qquad \forall v \in V. %\label{eq: bilin continuity}
%	\end{align*}
%	Moreover, there exists a $\sigma^* > 0$ such that for $\sigma \leq \sigma^* $, the operator $a$ is invertible with 
%	\begin{align*}
%	\eNorm {a_h^{-1} r} \leq C \HMinusOneDnorm r \qquad \forall r \in V_h %\label{eq: bilin stability}
%	\end{align*}
%	where $C$ is independent of $h$. 
%\end{theorem}
%\begin{proof}
%	The first estimate follows directly from Lemma \ref{la: SIPG continuous} and the definition of the norm \ref{def: h-1 seminorm}. If are able to show $a_h$ is coercive on $V_h$, we may apply classical existence and uniqueness theory, namely the Lax-Milgram theorem and thus deduce the inversibilty in $V_h$. As mentioned before we will use the energy norm $\eNormTwo{\cdot}$ to prove that $a_h$ is coercive. But due to their equivalence the coercivity of $a_h$ also holds with respect to the norm $\eNorm{\cdot}$.
%	
%	Since the diffusion matrix $A$ is positive definite it holds
%	\begin{align}
%	\int_\Omega \nabla w \cdot A \nabla w \geq \lambda \LTwonorm {\nabla w }^2  \label{eq: estimate pos def}
%	\end{align}
%	for a $\lambda >0$.
%	Thus, we have 
%	\begin{align}
%	\bilin {a_h(v)} {v}  = \bilinA {\nabla v} {\nabla v} - 2 \sum\limits_{e \in \edges}\myIntS e { \jump {v \average{A \nabla v} } + \sum_{e\in \edgesi} \frac \sigma {|e|} \int_e \jump v^2 .\label{eq: coerc first estimate}
%	\end{align}
%	We derived in \eqref{eq: CS estimate} the estimate 
%	\begin{align}
%	\sum\limits_{e \in \edges}\myIntS e { \jump {v \average{A \nabla v} }
%	\leq		\left( \sum\limits_{e \in \edges} \frac {\sigma}{|e|} \LTwonormE{\jump {v}}^2 \right)^{\frac 1 2}
%	\left( \sum\limits_{e \in \edges} \frac {|e|} \sigma \LTwonormE{\average{A \nabla v}}^2 \right)^{\frac 1 2} \label{eq: inter estimate}
%	\end{align}
%	Taking the root of the trivial identity $ab \leq \varepsilon^2 a^2 + 2ab + {\frac 1 \varepsilon}^2b^2 = \left( \varepsilon a + \frac 1 \varepsilon b\right)^2$ yields the inequality $\sqrt{a} \sqrt{b} \leq \varepsilon a + \frac 1 \varepsilon b$ for arbitrary $a,b \in R$ and $\varepsilon \geq 0$. Thus, we can derive from \eqref{eq: inter estimate}
%	\begin{align*}
%	\sum\limits_{e \in \edges}\myIntS e { \jump {v \average{A \nabla v} } \leq \frac 1 \varepsilon \sum\limits_{e \in \edges}\frac \sigma {|e|}\LTwonormE{\jump {v}}^2 
%	+ \varepsilon  \sum\limits_{e \in \edges}  \frac {|e|} \sigma \LTwonormE{\average{A \nabla v}}^2
%	\end{align*}
%	Using this estimate in \eqref{eq: coerc first estimate} we can derive
%	\begin{align*}
%	a_h(v,v)  \geq&
%	\bilinA {\nabla v} {\nabla v} 
%	-	 2 \varepsilon  \sum\limits_{e \in \edges}  \frac {|e|} \sigma \LTwonormE{\average{A \nabla v}}^2
%	+ \left(-\frac 1 \varepsilon +1\right) \sum_{e\in \edgesi}  \frac \sigma {|e|} \LTwonormE{\jump {v}}^2
%	\end{align*}
%	Subtracting $\alpha \eNorm v$ from both sides implies
%	\begin{align}
%	a_h(v,v) - \alpha  \eNorm v
%	\geq& (1-\alpha) \bilinA {\nabla v} {\nabla v} 
%	+ (-2 \varepsilon -\alpha) \sum\limits_{e \in \edges}  \frac {|e|} \sigma \LTwonormE{\average{A \nabla v}}^2 \nonumber\\
%	&+ \left(-\frac 1 \varepsilon +1-\alpha\right) \sum_{e\in \edgesi}  \frac \sigma {|e|} \LTwonormE{\jump {v}}^2.
%	\end{align}
%	If we can find values for $\sigma$ and $\varepsilon$ such that the right-hand side is positive we proved the coercivity. To further simplify the right-hand side we estimate the second term by something we can combine with the first summand. To start, we apply the trace estimate and we find 
%	\begin{align}
%	\sum_{e \in \edges} \singleNorm e \LTwonormE{\average{A \nabla v}} 
%	\leq C \sum_{T \in \triang} \left( \LTwonormT{\average{A \nabla v}} +\HOnenormT{\average{A \nabla v}}   \right).
%	\end{align}
%	Recalling the definition of the average (cf. Definition \ref{def: edge operators}) it is clear we can further simplify to
%	\begin{align}
%	\sum_{e \in \edges} \singleNorm e \LTwonormE{\average{A \nabla v}} 
%	\leq C \sum_{T \in \triang} \left( \LTwonormT{A \nabla v} +h^2 \HOnenormT{ A \nabla v}   \right).
%	\end{align}
%	By the inverse estimate from Lemma \ref{la: inverse estimate} we have 
%	\begin{align}
%	\sum_{e \in \edges} \singleNorm e \LTwonormE{\average{A \nabla v}} \leq C \LTwonormT{A \nabla v}
%	\end{align}
%	
%	
%	Let $\lambda (>0$) be the largest eigenvalue of $A$. Then it holds
%	\[
%	\LTwonormT{A \nabla v} = \myIntX  T { A \nabla v \cdot A\nabla v \leq \myIntX  T { \lambda \nabla v \cdot A\nabla v \leq \lambda \bilinA {\nabla v}{\nabla v}
%	\]	
%	where $C$ is a positive constant depending only on the triangulation and the diffusion matrix $A$ and $\lambda$ is also a positive constant (cf. \eqref{eq: estimate pos def}).
%	Choosing $0 < \varepsilon < \frac \lambda C$ and $\sigma^* > \frac 1 \varepsilon$ then implies that 
%	the bilinear from $a$ is coercive. 
%\end{proof}
%Note that this proof depends on a careful choice of $\sigma$. If the diffusion matrix $A$ has only small eigenvalues the penalty parameter has to be large. But a large penalty parameter yields to a larger condition number of the stiffness matrix.
%
%To conclude our analysis of the SIPG method we derive a statement similar to C\'eas Lemma. The proof was taken from \cite[Lemma 10.5.2]{BS2002} and fitted to our context.
%\begin{theorem}[Approximation Properties]\label{thm: error estimate}
%	Let $u$ be the solution of the General Poisson Problem \ref{def: General Poisson Problem} and $\sigma \geq \sigma^*$ of Theorem \ref{thm: SIPG stability}. Let $u_h \in V_h$ satisfy \eqref{eq: DG system}, namely
%	\[
%	a_h(u_h, v_h) = l(v_h) + J^\sigma_0(v_h)  \qquad \forall v_h \in V_h.
%	\]
%\end{theorem}
%
%
%Then there exists a positive constant $C$ independent of $h$ such that 
%\[
%	\eNorm{u - u_h } \leq \frac C \alpha \inf_{v \in v_h} \eNorm{u - v_h}
%\]
%where $\alpha$ is the coercivity constant of $a_h$ with respect to $\eNorm{\cdot}$.
%
%\begin{proof}
%	Let $v_h$ be in $V_h$.
%The bilinear form $a$ is coercive on $V_h$, let $\alpha$ be the corresponding coercive constant. 
%Thus, it holds
%\[
%	a_h(v_h, w_h) \geq \alpha \eNorm{v_h} \eNorm{w_h} 
%	\; \Leftrightarrow \;
%	\eNorm{v_h} \leq \frac 1 \alpha \frac {a_h(v_h,w_h)} {\eNorm{w_h}} 	
%\]
%Therefore, we have
%\begin{align*}
%  \eNorm{u-u_h} \leq& \eNorm{u-v_h} + \eNorm{u_h-v_h} \\
%%  \leq& \eNorm{u-v_h} + \frac 1  \alpha \sqrt{a_h(u_h-v_h, u_h-v_h)} \\
%  \leq& \eNorm{u-v_h} + \frac 1 \alpha \sup_{w \in V_h, w \neq 0} \frac {a_h(u_h-v_h, w_h) }{\eNorm{w_h}}
%\end{align*}
%Since both $u$ and $u$ satisfy
%\[
%	a_h(u,v_h) = l(v_h) = a_h(u_h,v_h) \qquad \forall v_h \in V_h
%\]
%it holds
%\begin{align*}
%  \eNorm{u-u_h} \leq& \eNorm{u-v_h} + \frac 1 \alpha \sup_{w \in V_h, w \neq 0} \frac {a(u-v_h, w_h) }{\eNorm{w_h}}
%\end{align*}
%By Theorem \ref{thm: SIPG stability} it then follows that
%\begin{align*}
%\eNorm{u-u_h} \leq& \eNorm{u-v_h} + \frac C \alpha \eNorm{u-v_h} \leq \frac C \alpha \eNorm{u-v_h}
%\end{align*}
%Since $v_h$ was arbitrary this implies the claim.
%\end{proof}


\section{Well-posedness of the SIPG Method}

In this section we want to prove that the derived SIPG method is well-posed. The proofs were taken from \cite{BS2002, PPO+2000} and extended to the case of a Generalised Poisson Problem. 
To start we first have to define the norm in which the analysis will be carried out with. Let $V$ be $L^2(\Omega) \cup H^1(\Omega; \triang)$ 
\begin{definition}[Energy Norm] \label{def: energy norm}
	The mesh-dependent energy norm for a function $v \in V$ is defined by
	\[
	\eNorm v ^2 := \LTwonorm{\nabla v}^2 + \frac 1 \sigma \sum_{e \in \edges} |e|\LTwonormE{\average{\nabla v}}^2 + 2 \sigma \sum_{e \in \edges} \frac 1 {|e|}\LTwonormE{\jump{v}}^2.
	\]
\end{definition}

Note that through this section $C$ denotes a generic positive constant independent of $h$ that may take different values in each equation.

At first, we show the following statement
\begin{lemma}[Boundedness of the SIPG method]\label{la: SIPG continuous}
	For the SIPG method it holds
	\[
	|a_h^{DG}(\varphi, v) | \leq C \eNorm{\varphi} \eNorm{v} \qquad \forall \varphi,v \in V
	\]
	for a positive constant $C$ depending only on $A$.
\end{lemma}
\begin{proof}
	Let us consider the terms of \eqref{eq:inner product SIPG} and \eqref{eq: penalty term} separately.
	For the first term we have by the Cauchy-Schwarz inequality
	\begin{align}
	\singleNorm{\sum\limits_{T \in \triang} \myIntX  T { \nabla \varphi \cdot A \nabla v}}
	 =& \singleNorm{\bilinLTwo {\nabla \varphi} {\nabla v} }
	  \leq \LTwonorm{\nabla \varphi} \LTwonorm{A} \LTwonorm{\nabla v}. \label{eq: CS estimate 0}
	\end{align}
	Considering the sum of the second and the fourth term we find using Cauchy-Schwarz and using also the discrete Cauchy-Schwarz inequality
	\begin{align}
	\singleNorm{\sum\limits_{e \in \edges}\myIntS e { \jump {\varphi \average{A \nabla v} }}} &\leq
	\sum\limits_{e \in \edges}  \LTwonormE{\jump {\varphi}} \LTwonormE {\average{A \nabla v}} \nonumber \\
	& \leq
	\left( \sum\limits_{e \in \edges} \frac {\sigma}{|e|} \LTwonormE{\jump {\varphi}}^2 \right)^{\frac 1 2}
	\left( \sum\limits_{e \in \edges} \frac {|e|} \sigma \LTwonormE{\average{A \nabla v}}^2 \right)^{\frac 1 2} 	 \nonumber \\
	& \leq
	\left( \sum\limits_{e \in \edges} \frac {\sigma}{|e|} \LTwonormE{\jump {\varphi}}^2 \right)^{\frac 1 2}
	 \left( \sup\limits_{e \in \edges} \LTwonormE{\average{A}}\sum\limits_{e \in \edges} \frac {|e|} \sigma \LTwonormE{\average{\nabla v}}^2 \right)^{\frac 1 2} \label{eq: CS estimate}
	\end{align}
	For shorter notation we abbreviate $\sup\limits_{e \in \edges} \LTwonormE{\average{A}}$ by $\bar A$. Since the third and the last term in \eqref{eq:inner product SIPG} are the symmetric to the second and fourth summands, we analogously have for their sum
	\begin{align}
	\singleNorm{\sum\limits_{e \in \edges}\myIntS e { \jump {v \average{A \nabla \varphi} }}} \leq&
	\left( \sum\limits_{e \in \edges} \frac {\sigma}{|e|}\LTwonormE{\jump {v}}^2 \right)^{\frac 1 2}
	\left( \bar A \sum\limits_{e \in \edges} \frac {|e|} \sigma \LTwonormE{\average{\nabla \varphi}}^2 \right)^{\frac 1 2}.\label{eq: CS estimate2}
	\end{align}
	Now only the penalty term (cf. \eqref{eq: penalty term}) needs to be estimated. Again we apply at first the Cauchy-Schwarz inequality and then the discrete Cauchy-Schwarz inequality:
	\begin{align}
	\singleNorm {\sum_{e \in \edges} \frac \sigma {|e|} \myIntS e {\jump \varphi \jump v }}
	\leq& \sum_{e \in \edges} \frac \sigma  {|e|} \myIntS{e}{ \singleNorm {\jump \varphi \jump v }} \nonumber \\
	\leq& \left( \sum\limits_{e \in \edges} \frac \sigma{|e|} \LTwonormE{\jump{\varphi}}^2 \right)^{\frac 1 2} \left( \sum\limits_{e \in \edges} \frac \sigma {|e|} \LTwonormE{\jump{v}}^2 \right)^{\frac 1 2} \label{eq: CS estimate3}
	\end{align}
	Adding \eqref{eq: CS estimate 0}, \eqref{eq: CS estimate}, \eqref{eq: CS estimate2} and \eqref{eq: CS estimate3} we find using the discrete Schwarz inequality once more
	\begin{align}
	|a_h^{DG}(\varphi,v)| \leq& 
	 \LTwonorm{A} \LTwonorm {\nabla \varphi} \LTwonorm{\nabla v} \nonumber\\
	& +\left( \sum\limits_{e \in \edges}\frac {\sigma}{|e|}\LTwonormE{\jump {\varphi}}^2 \right)^{\frac 1 2}
	\left( \bar A \sum\limits_{e \in \edges} \frac {|e|} \sigma \LTwonormE{\average{\nabla v}}^2 \right)^{\frac 1 2} \nonumber\\
	& +	\left( \sum\limits_{e \in \edges} \frac {\sigma} {|e|}\LTwonormE{\jump {v}}^2 \right)^{\frac 1 2}
	\left( \bar A \sum\limits_{e \in \edges} \frac {|e|} \sigma \LTwonormE{\average{\nabla \varphi}}^2 \right)^{\frac 1 2}\nonumber \\
	& + \left( \sum\limits_{e \in \edges} \frac \sigma{|e|} \LTwonormE{\jump{\varphi}}^2 \right)^{\frac 1 2} \left( \sum\limits_{e \in \edges} \frac \sigma {|e|} \LTwonormE{\jump{v}}^2 \right)^{\frac 1 2} \nonumber \\
	\leq& 
	\left( 
	\LTwonorm{A}\LTwonorm{\nabla \varphi}^2+2\sum\limits_{e \in \edges} \frac {\sigma} {|e|}\LTwonormE{\jump {\varphi}}^2+ \bar A\sum\limits_{e \in \edges} \frac {|e|} \sigma \LTwonormE{\average{\nabla \varphi}}^2
	\right)^{\frac 1 2} \nonumber \\
	&\times
	\left( 
	\LTwonorm{A} \LTwonorm{\nabla v}^2+2\sum\limits_{e \in \edges} \frac {\sigma}{|e|}\LTwonormE{\jump {v}}^2+ \bar A\sum\limits_{e \in \edges} \frac {|e|} \sigma \LTwonormE{\average{\nabla v}}^2
	\right) ^{\frac 1 2} \nonumber \\
	\leq & \max {\LTwonorm{A}} {\sup_{e \in \edges }\LTwonormE{\average A}} \eNorm \varphi \eNorm v   \label{eq: exact estimate boundedness} % \leq \eNorm{A} \eNorm{\varphi}
	\end{align}
	\phantom{blub}
\end{proof}

The next step in our analysis considers the stability of the SIPG method. To do we first need to define a norm measuring elements of $V'$.
\begin{definition} \label{def: h-1 seminorm}
	The semi-norm $\HMinusOneDnorm \cdot $ of an element $r \in V_h'$ is given by 
	\[
	\HMinusOneDnorm r  := \sup_{0 \neq w \in V_h} \frac {{\bilin r w } } {\eNorm{w}}.
	\]
\end{definition}
Before we can start we need to define a second mesh-dependent energy norm which simplifies the proof that the bilinear form in the SIPG method is coercive.
\begin{definition}[Energy Norm]\label{energy norm2}
	The mesh-dependent discrete energy norm $\eNormTwo \cdot $ is defined by
	\[
	\eNormTwo v^2 :=  \LTwonorm v^2 +  \sigma \sum_{e \in \edges} \frac 1 {|e|}\LTwonormE{\jump{\nabla v}}^2.
	\]
\end{definition}
Calling this norm also an energy norm is justified as both norms are equivalent on $V_h$.
\begin{lemma}[Equivalence of Energy Norms in $V_h$] \label{la: equivalence energy norm}
	For any $w \in V_h$ holds
	\[
	c \eNormTwo w \leq \eNorm w \leq C(1+\frac  1 \sigma)\eNormTwo w
	\]
	with positive Constants $c, C$ depending only on the mesh.
\end{lemma}
\begin{proof}
	By definition the inequality $c \eNormTwo w \leq \eNorm w$ is clear and holds for example for $c=1$. 
	By the trace estimate from Lemma \ref{la: trace estimate} we have
	\begin{align*}
	\sum_{e \in \edges} \singleNorm e \LTwonormE{\average{\nabla w}}^2 
	\leq C \sum_{T \in \triang} \left( \LTwonormT{\average{\nabla w}}^2 +\HOnenormT{\average{\nabla w}}^2   \right).
	\end{align*}
	Recalling the definition of the average (cf. Definition \ref{def: edge operators}) it is clear we can further simplify to
	\begin{align*}
	\sum_{e \in \edges} \singleNorm e \LTwonormE{\average{\nabla w}}^2 
	\leq C \sum_{T \in \triang} \left( \LTwonormT{\nabla w}^2 +h^2 \HOnenormT{ \nabla w}^2   \right).
	\end{align*}
	By the inverse estimate from Lemma \ref{la: inverse estimate} we have 
	\begin{align}
	\sum_{e \in \edges} \singleNorm e  \LTwonormE{\average{\nabla w}}^2 \leq C \LTwonorm{\nabla w}^2. \label{eq: estimate average}
	\end{align}
	Therefore it can be derived
	\begin{align*}
		\eNorm{w}^2 
		&= \LTwonorm{\nabla w}^2
			+ \frac 1 \sigma \sum_{e \in \edges} \singleNorm e \LTwonormE{\average{\nabla w}}^2  
			+ 2 \sigma \sum_{e \in \edges} \frac 1 {|e|} \LTwonormE{\jump{w}}^2 \\
		&\leq (1+\frac C \sigma) \LTwonorm{\nabla w}^2 
			+ 2 \sigma \sum_{e \in \edges} \frac 1 {|e|} \LTwonormE{\jump{w}}^2 
		\leq C(1+\frac 1 \sigma) \eNormTwo{w}^2.
	\end{align*}
The result follows from the monotonicity of the square root and the fact $a \geq \sqrt a$ for all $a > 1$.
\end{proof}

\begin{theorem}[Stability]\label{thm: SIPG stability}
	For $\mathcal A_h:V \rightarrow V'$ defined by 
	\[
	\bilin {\mathcal A_h v} \varphi := a_h^{DG}(\varphi, v) \qquad \forall \varphi \in V
	\]
	it holds that  
	\begin{align*}
	\HMinusOneDnorm{\mathcal A_h v} \leq C \eNorm v \qquad \forall v \in V. %\label{eq: bilin continuity}
	\end{align*}
	for a positive constant $C$ depending only on the mesh and the diffusion matrix $A$. 
	
	Further there exists $\alpha > 0 $ such that for all $v_h \in V_h$
	\begin{align}
		a_h^{DG}(v_h, v_h) \geq \alpha \eNorm{v_h}^2 \label{eq: coercivity DG form}
	\end{align}	
	Moreover, this implies the existence of $\sigma^* > 0$ such that for $\sigma \leq \sigma^* $, the operator $a_h$ is invertible with 
	\begin{align*}
	\eNorm {\mathcal A_h^{-1} {r_h}} \leq C \HMinusOneDnorm {r_h} \qquad \forall r_h \in V_h %\label{eq: bilin stability}
	\end{align*}
	where $C$ is independent of $h$. 
\end{theorem}
\begin{proof}
	The first estimate follows directly from Lemma \ref{la: SIPG continuous} and the definition of the norm \ref{def: h-1 seminorm}. 
		
	Since the diffusion matrix $A$ is positive definite it holds for any $w \in V$
	\begin{align}
	\myIntX \Omega {\nabla w \cdot A \nabla w} \geq \lambda \LTwonorm {\nabla w }^2  \label{eq: estimate pos def}
	\end{align}
	for $\lambda >0$, the smallest eigenvalue of $A$.
	Therefore we find
	\begin{align}
	a_h^{DG}(v_h,v_h) \geq \lambda \LTwonorm{\nabla v_h}^2 - 2 \sum\limits_{e \in \edges}\myIntS e { \jump {v \average{A \nabla v_h} }} + \sum_{e\in \edgesi} \frac \sigma {|e|} \myIntS e {\jump {v_h}^2} .\label{eq: coerc first estimate}
	\end{align}
	We derived in \eqref{eq: CS estimate} the estimate 
	\begin{align}
	\sum\limits_{e \in \edges}\myIntS e { \jump {v_h \average{A \nabla v_h} }}
	\leq		\left( \sum\limits_{e \in \edges} \frac {\sigma}{|e|} \LTwonormE{\jump {v_h}}^2 \right)^{\frac 1 2}
	\left(C \sum\limits_{e \in \edges} \frac {|e|} \sigma \LTwonormE{\average{\nabla v_h}}^2 \right)^{\frac 1 2}. \label{eq: inter estimate}
	\end{align}
	Taking the root of the trivial identity $ab \leq \varepsilon^2 a^2 + 2ab + {\frac 1 \varepsilon}^2b^2 = \left( \varepsilon a + \frac 1 \varepsilon b\right)^2$ yields the inequality $\sqrt{a} \sqrt{b} \leq \varepsilon a + \frac 1 \varepsilon b$ for arbitrary $a,b \in R$ and $\varepsilon \geq 0$. Thus, we have using \eqref{eq: inter estimate}
	\begin{align*}
	\sum\limits_{e \in \edges}\myIntS e { \jump {v_h \average{A \nabla v_h} }} 
	\leq \frac 1 \varepsilon \sum\limits_{e \in \edges}\frac \sigma {|e|}\LTwonormE{\jump {v_h}}^2 
	+ C\varepsilon  \sum\limits_{e \in \edges}  \frac {|e|} \sigma \LTwonormE{\average{\nabla v_h}}^2.
	\end{align*}
	By \eqref{eq: estimate average} it follows that 
	\begin{align*}
		\sum\limits_{e \in \edges}\myIntS e { \jump {v_h \average{A \nabla v_h} }} 
		\leq \frac 1 \varepsilon \sum\limits_{e \in \edges}\frac \sigma {|e|}\LTwonormE{\jump {v_h}}^2 
		+ C\varepsilon  \frac C \sigma \LTwonorm{\nabla v_h}^2.
	\end{align*}			
	Using this estimate in \eqref{eq: coerc first estimate} and we can derive
	\begin{align*}
	a_h^{DG}(v_h,v_h)  \geq& (\lambda- \varepsilon \frac C \sigma)
	\LTwonorm{\nabla v_h}^2
	+ \left(-\frac 1 \varepsilon +1\right) \sum_{e\in \edgesi}  \frac \sigma {|e|} \LTwonormE{\jump {v_h}}^2.
	\end{align*}
	If we can find values for $\sigma$ and $\varepsilon$ such that $\lambda-\varepsilon \frac C \sigma$ and $-\frac 1 \varepsilon +1$ are positive we can estimate the right-hand side by $\min{\{\lambda-\varepsilon \frac C \sigma,-\frac 1 \varepsilon +1\}} \eNormTwo{v}^2$ and proved the desired coercivity. 
	Clearly, taking $\varepsilon$ to be larger than one ensures that the second term is positive. The first term is also positive if we choose $\sigma$ to be larger than $\frac {\varepsilon C} \lambda$.
	
	Now, that we have seen that $a_h^{DG}$ is coercive on $V_h$, we may apply classical existence and uniqueness theory, namely the Lax-Milgram theorem and thus deduce the inversibilty in $V_h$.
\end{proof}
Note that this proof depends highly on the smallest eigenvalue of the diffusion matrix $A$. If it has a very small eigenvalues also the coercivity constant $\alpha$ is small.
Additionally a small eigenvalue forces the penalty parameter to be chosen large. But a large penalty parameter yields to a larger condition number of the stiffness matrix.

To conclude our analysis of the SIPG method we derive a statement similar to C\'eas Lemma. The proof was taken from \cite[Lemma 10.5.2]{BS2002} and fitted to our context.
\begin{theorem}[Approximation Properties]\label{thm: error estimate}
	Let $u$ be the exact solution of the General Poisson Problem (cf. Definition \ref{def: General Poisson Problem}) and $\sigma \geq \sigma^*$ such that it satisfy the requirements of Theorem \ref{thm: SIPG stability}. Let $u_h \in V_h$ satisfy \eqref{eq: DG system}, namely
	\[
	a_h^{DG}(v_h, u_h) = l^{DG}_h(v_h) \qquad \forall v_h \in V_h.
	\]
\end{theorem}
Then there exists a positive constant $C$ independent of $h$ such that 
\[
\eNorm{u - u_h } \leq (1+\frac C \alpha) \inf_{v \in v_h} \eNorm{u - v_h}
\]
where $\alpha$ is the coercivity constant of \eqref{eq: coercivity DG form}. % and $\frac C \alpha > 1$ holds.

\begin{proof}
	Let $v_h$ be in $V_h$.
	The bilinear form $a$ is coercive on $V_h$, let $\alpha$ be the corresponding coercive constant. 
	Then it holds
	\[
		\eNorm{v_h} \leq \frac 1 \alpha \frac {a_h^{DG}(v_h, v_h)} {\eNorm{v_h}} 
		 \leq \frac 1 \alpha \sup_{w_h \in V_h, w_h \neq 0} \frac {a_h^{DG}(v_h,w_h)} {\eNorm{w_h}}.
	\]
	Therefore, we have
	\begin{align*}
	\eNorm{u-u_h} \leq& \eNorm{u-v_h} + \eNorm{u_h-v_h} \\
	%  \leq& \eNorm{u-v_h} + \frac 1  \alpha \sqrt{a_h(u_h-v_h, u_h-v_h)} \\
	\leq& \eNorm{u-v_h} + \frac 1 \alpha \sup_{w_h \in V_h, w_h \neq 0} \frac {a_h^{DG}(u_h-v_h, w_h) }{\eNorm{w_h}}.
	\end{align*}
	Since both $u$ and $u_h$ satisfy
	\[
	a_h^{DG}(u,v_h) = l_h^{DG}(v_h) = a_h^{DG}(u_h,v_h) \qquad \forall v_h \in V_h
	\]
	it holds that
	\begin{align*}
	\eNorm{u-u_h} \leq& \eNorm{u-v_h} + \frac 1 \alpha \sup_{w \in V_h, w \neq 0} \frac {a_h^{DG}(u-v_h, w_h) }{\eNorm{w_h}}.
	\end{align*}
	By Theorem \ref{thm: SIPG stability} it then follows that
	\begin{align*}
	\eNorm{u-u_h} \leq& \eNorm{u-v_h} + \frac C \alpha \eNorm{u-v_h}.
	\end{align*}
	Since $v_h$ is arbitrary this implies the claim.
\end{proof}


