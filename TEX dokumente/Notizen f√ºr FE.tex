\documentclass[a4paper,11pt]{article}
\usepackage[T1]{fontenc}
\usepackage[utf8]{inputenc}
\usepackage{lmodern}
\usepackage[T1]{fontenc}
\usepackage[utf8]{inputenc}
\usepackage{lmodern}
\usepackage{ngerman}
\usepackage{cite}
\usepackage{amssymb}
\usepackage{amsfonts}
\usepackage{amsmath}
\usepackage{stmaryrd}

\newcommand{\D}{\operatorname{D}}


\title{Notizen für Finite Elemente Verfahren}
\author{Elisa}


\begin{document}

\maketitle

There are three type of important cells: reference cell, base cell und leaf cell.

\section*{Affine Transformation}
For every base cell exist a affine transformation $\Phi(x) = Ax+B$ such that $J($reference cell$)$=base cell. It should hold that 

\[
\Phi\left(\begin{pmatrix} 0 \\ 0 \end{pmatrix}\right) = P_0, \Phi\left(\begin{pmatrix} 0 \\ 1 \end{pmatrix}\right) = P_1 \textnormal{ and } \Phi\left(\begin{pmatrix} 1 \\ 0 \end{pmatrix}\right) = P_2
\]
Thus $ A = \begin{pmatrix} P_1-P_0 & P_2-P_0\end{pmatrix}$ and $B = P_0$

\section*{Gradient}

\subsection*{From reference cell to base cell}
For the gradient $\nabla u$ holds:
\[\nabla u  = J^{-t} \nabla \hat u \]
where $J$ is the Jacobian of $\Phi$ (i.e. $A$) and $\nabla \hat u$ is the gradient in the reference cell.

\subsection*{From base cell to leaf cell}
To transform a basecell into its child on the $l$th level, one can also use an affine transformation $\Psi(x) = A'x+ B'$. $A'$ just performs a scaling and therefore is of the form 
\[
A' = \begin{pmatrix}  \frac 1  {2^l} & 0 \\ 0 & \frac 1 {2^l}\end{pmatrix} \text{ or } A' = \begin{pmatrix}  -\frac 1  {2^l} & 0 \\ 0 & -\frac 1 {2^l}\end{pmatrix}
\]

(the negative sign comes from the rotation of the 0 child) and

\[
A'^{-t} = \begin{pmatrix}  {2^l} & 0 \\ 0 & {2^l}\end{pmatrix} \text{ resp. } A'^{-t} = \begin{pmatrix}  -{2^l} & 0 \\ 0 & -{2^l}\end{pmatrix}
\]


Thus, 
\[\nabla u'  = A'^{-t} \nabla u  = 2^l \nabla u \text { resp. } -2^l \nabla u\]

\section*{Hessian}

\subsection*{From reference cell to base cell}
For the Hessian $\D^2 u$ holds:
\[\D^2u = J^{-t} \D^2\hat u J^{-1}\]
where $J$ is the Jacobian of $\Phi$ (i.e. $A$) and $\D^2\hat u$ is the Hessian in the reference cell.

\subsection*{From base cell to leaf cell}
It holds:
\[\D^2u' = A'^{-t} \D^2 u A'^{-1} = 2^{2l} \D^2 u\]

Note, that this also holds for child 0. The minus signs cancel because of the multiplication from the left and the right.

\subsection*{What happens if the diffusion is small}
Consider
\[
	\nabla \cdot (A \cdot \nabla u) = f
\]
First consider the 1d case
\begin{align*}
	&\frac {d} {dx} a \frac {d} {dx}  u = f \\
	\Leftrightarrow &\; a \frac {d^2}{dx^2} u = f	\\
	\Leftrightarrow &\; u = \int \int \frac f	a\\
\end{align*}
Thus, for small $a$ the solution increases a lot and for a large $a$ it is really small.
\end{document}
