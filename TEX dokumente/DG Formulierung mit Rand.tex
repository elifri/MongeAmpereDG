\documentclass[a4paper,11pt]{article}
\usepackage[T1]{fontenc}
\usepackage[utf8]{inputenc}
\usepackage{lmodern}
\usepackage{ngerman}
\usepackage{cite}
\usepackage{amssymb}
\usepackage{amsfonts}
\usepackage{amsmath}
\usepackage{stmaryrd}


\title{atuelle Implementierung}

\newcommand{\myint}{\displaystyle\int}
\newcommand{\cof}{\operatorname{cof}}
\newcommand{\edges}{\mathcal{E}_h}
\newcommand{\edgesi}{\mathcal{E}_h^i}
\newcommand{\edgesb}{\mathcal{E}_h^b}
\newcommand{\triang}{\mathcal{T}_h}

\begin{document}
\maketitle

Consider the following PDE 
\begin{equation}
	\frac 1 2 (w_{yy} v_{xx}+w_{xx} v_{yy} - w_{yx} v_{xy} - w_{xy} v_{yx}) = f \qquad \textnormal{ on } \Omega \label{PDE}		
\end{equation}

$\begin{pmatrix} w_{yy} & -w_{xy} \\ -w_{yx} & w_{xx} \end{pmatrix}$ is the cofactor matrix of the Hessian of $w$ and is from now on denoted by $\cof(D^2 w)$. 
To abbreviate $\left( w_{yyx}-w_{yxy}, -w_{xyx} + w_{xxy} \right)^t$ we use the term  $\nabla \cdot \cof(D^2 w)$.
Thus, we can express the left-hand side of the weak formulation by the following bilinear form

\[
	a(v,\varphi) = \myint_{\Omega} -\nabla \varphi \cdot \cof(D^2 w) \nabla v + \myint_{\partial \Omega} \varphi\cof(D^2 w) \nabla v \cdot n 
	-  \myint_\Omega \varphi (\nabla \cdot \cof(D^2 w)) \cdot \nabla v
\]

Let $\mathcal{T}_h$ be shape regular, uniform triangulation of $\Omega$ and denote its edge set by $\mathcal{E}_h$. We divide $\mathcal E _h $ into the disjoint subsets $\mathcal{E}_h^i$ which contains the inner edges and $\mathcal{E}_h^b$ which contains the boundary edges. \\
In addition let $\mathcal V_h$ be the finite element space of piecewise quadratic polynomials, namely
\[
	\mathcal V_h = \{ f \in H^1(\Omega); f \arrowvert_T \textnormal{ is a polynomial of (total) degree} = 2 \textnormal{ for every } t \in \mathcal T_h\}
\]
Futhermore let $w$ be $\mathcal V_h$ such that we can define the Hessian of $w$ on every element and let us assume $\varphi \in H^2(\Omega; \triang) \cap H^1(\Omega)$.
 Decomposing our integral in the contributions of each element we gain

\begin{eqnarray*}
a(v, \varphi) &=& \sum_{T \in \triang} \myint_T -\nabla \varphi \cdot \cof(D^2 w) \nabla v + \sum_{T \in \triang}\myint_{\partial T} \varphi \cof(D^2 w) \nabla v \cdot n \\
	&&-  \sum_{T \in \triang} \myint_T \varphi (\nabla \cdot \cof(D^2 w)) \cdot \nabla v
\end{eqnarray*}

If we take a closer look on $\nabla \cdot \cof(D^2 w) = \left( w_{yyx}-w_{yxy}, -w_{xyx} + w_{xxy} \right)^t$ we discover that it equals zero because of Schwarz' theorem. Thus, the last term in $a$ vanishes.

Since two adjacent elements share one edge with opposite normal vectors we can rewrite the middle term by

\begin{eqnarray*}
 a(u,v) = & &\sum\limits_{T \in \triang} \myint_T -\nabla v \cdot \cof(D^2 w) \nabla u \\
	 &+ &\sum\limits_{e \in \edgesi}\myint_{e} \left( \{\{\varphi\}\} \llbracket \cof(D^2 w) \nabla v \cdot n \rrbracket + \llbracket\varphi\rrbracket \{\{ \cof(D^2 w) \nabla v \cdot n \}\} \right)
\end{eqnarray*}

and

\begin{eqnarray*}
	f(u,v) = && \sum\limits_{T \in \triang} \myint_T v f \\
	 				&- &\sum\limits_{e \in \edgesb}\myint_{e} v \cof(D^2 w) \nabla u \cdot n 
\end{eqnarray*} 

Symmetrising our bilinear form $a$ we get

\begin{eqnarray*}
 a_S(v, \varphi) = &\sum\limits_{T \in \triang} \myint_T -\nabla \varphi \cdot \cof(D^2 w) \nabla v \\
 + &\sum\limits_{e \in \edgesi}\myint_{e} \left( \{\{\varphi\}\} \llbracket \cof(D^2 w) \nabla v \cdot n \rrbracket + \llbracket\varphi\rrbracket \{\{ \cof(D^2 w) \nabla v \cdot n \}\} \right)  \\
 +& \sum\limits_{e \in \edgesi}\myint_{e} \left( \{\{v\}\} \llbracket \cof(D^2 w) \nabla \varphi \cdot n \rrbracket + \llbracket v\rrbracket \{\{ \cof(D^2 w) \nabla \varphi \cdot n \}\} \right) 
\end{eqnarray*}

Neglecting terms near zero we have

\begin{eqnarray*}
 a_S(v, \varphi) = &\sum\limits_{T \in \triang} \myint_T -\nabla \varphi \cdot \cof(D^2 w) \nabla v \\
 + &\sum\limits_{e \in \edgesi}\myint_{e} \llbracket\varphi\rrbracket \{\{ \cof(D^2 w) \nabla v \cdot n \}\} \\
 +& \sum\limits_{e \in \edgesi}\myint_{e} \llbracket v\rrbracket \{\{ \cof(D^2 w) \nabla \varphi \cdot n \}\} 
\end{eqnarray*}


To enforce stability we enforce the following penalty term [TICAM report 3.2.2.]

\begin{eqnarray*}
	J^\sigma(\varphi, v) = \sum\limits_{e \in \edges} \myint_e \frac \sigma {|e|} \llbracket \varphi \rrbracket \llbracket v \rrbracket
\end{eqnarray*}

Thus, we end up with the problem finding $v \in V_h \subset H^1(\Omega)$ such that
\[
	a_S(\phi,v) + J^\sigma(\varphi,v) = 2f(\varphi, v) \textnormal{for all } \varphi \in H^2(\Omega; \triang) \cap H^1(\Omega) 
\] 
\end{document}
