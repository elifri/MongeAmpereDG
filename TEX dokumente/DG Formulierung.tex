\documentclass[a4paper,11pt]{article}
\usepackage[T1]{fontenc}
\usepackage[utf8]{inputenc}
\usepackage{lmodern}
\usepackage{ngerman}
\usepackage{cite}
\usepackage{amssymb}
\usepackage{amsfonts}
\usepackage{amsmath}
\usepackage{stmaryrd}


\title{DG formulation}

\newcommand{\myint}{\displaystyle\int}
\newcommand{\cof}{\operatorname{cof}}
\newcommand{\edges}{\mathcal{E}_h}
\newcommand{\edgesi}{\mathcal{E}_h^i}
\newcommand{\edgesb}{\mathcal{E}_h^b}
\newcommand{\triang}{\mathcal{T}_h}

\begin{document}
\maketitle

Consider the following PDE 
\begin{equation}
	\frac 1 2 (w_{yy} v_{xx}+w_{xx} v_{yy} - w_{yx} v_{xy} - w_{xy} v_{yx}) = f \qquad \textnormal{ on } \Omega \label{PDE}		
\end{equation}
with dirichlet boundary condition $v = u_0$.

At first, we multiply it by $-1$
\begin{equation}
	\frac 1 2 (-w_{yy} v_{xx}-w_{xx} v_{yy} + w_{yx} v_{xy} + w_{xy} v_{yx}) = -f \qquad \textnormal{ on } \Omega \label{PDE}		
\end{equation}

%\begin{eqnarray*}
%	&\myint_{\Omega}& \varphi (w_{yy} v_{xx}+w_{xx} v_{yy}) \\
%	= & \myint_{\Omega}& \varphi (\partial_x, \partial_y) \cdot \left( \begin{pmatrix} w_{yy} v_x \\ w_{xx} v_y \end{pmatrix} \right) - \myint_\Omega \varphi ( w_{yyx}v_x + w_{xxy}v_y ) \\ 
%	= & \myint_{\Omega}& \varphi (\partial_x, \partial_y) \cdot \left( \begin{pmatrix} w_{yy} & 0 \\ 0 & w_{xx} \end{pmatrix} \nabla v \right) - \myint_\Omega \varphi \begin{pmatrix} w_{yyx} \\ w_{xxy} \end{pmatrix} \times \nabla v\\ 
%	= & \myint_{\Omega}& -\nabla \varphi^T \begin{pmatrix} w_{yy} & 0 \\ 0 & w_{xx} \end{pmatrix} \nabla v + \myint_{\partial \Omega} \varphi \begin{pmatrix} w_{yy} & 0 \\ 0 & w_{xx} \end{pmatrix} \nabla v \cdot n - \myint_\Omega \varphi \begin{pmatrix} w_{yyx} \\ w_{xxy} \end{pmatrix} \times \nabla v
%\end{eqnarray*}

%Mit dem Rest der Gleichung folgt dann

It holds for the left-hand side

\begin{eqnarray*}
	&\myint_{\Omega}& \varphi (-w_{yy} v_{xx}-w_{xx} v_{yy}+w_{yx} v_{xy} + w_{xy} v_{yx}) \\
	= && -\myint_{\Omega} \varphi \: \nabla \cdot \begin{pmatrix} w_{yy} v_x \\ w_{xx} v_y \end{pmatrix} + \myint_\Omega \varphi (w_{yyx}v_x + w_{xxy}v_y) \\
	&+&\myint_{\Omega} \varphi \: \nabla \cdot \begin{pmatrix} w_{xy} v_y \\ w_{yx} v_x \end{pmatrix}  - \myint_\Omega \varphi  (w_{xyx}v_y +  w_{yxy}v_x ) \\ 
	= && -\myint_{\Omega} \varphi\: \nabla \cdot \left( \begin{pmatrix} w_{yy} & 0 \\ 0 & w_{xx} \end{pmatrix} \nabla v \right) + \myint_\Omega \varphi \left(w_{yyx} , w_{xxy} \right) \cdot \nabla v\\ 
	&+ & \myint_{\Omega} \varphi\: \nabla \cdot \left( \begin{pmatrix} 0 & w_{xy}  \\ w_{yx} & 0 \end{pmatrix} \nabla v \right) - \myint_\Omega \varphi \left(w_{yxy} , w_{xyx} \right) \cdot \nabla v\\ 
	= && \myint_{\Omega} \nabla \varphi \cdot \begin{pmatrix} w_{yy} & -w_{xy} \\ -w_{yx} & w_{xx} \end{pmatrix} \nabla v - \myint_{\partial \Omega} \varphi \begin{pmatrix} w_{yy} & -w_{xy} \\ -w_{yx} & w_{xx} \end{pmatrix} \nabla v \cdot n \\
	&+ & \myint_\Omega \varphi \left( w_{yyx}-w_{yxy}, -w_{xyx} + w_{xxy} \right) \cdot \nabla v
\end{eqnarray*}

%Das heißt, wir können die linke Seite durch folgende Bilinearform ausdrücken
$\begin{pmatrix} w_{yy} & -w_{xy} \\ -w_{yx} & w_{xx} \end{pmatrix}$ is the cofactor matrix of the Hessian of $w$ and is from now on denoted by $\cof(D^2 w)$. 
To abbreviate $\left( w_{yyx}-w_{yxy}, -w_{xyx} + w_{xxy} \right)^t$ we use the term  $\nabla \cdot \cof(D^2 w)$.
Thus, we can express the left-hand side of the weak formulation by the following bilinear form

%\[
%	a(u,v) = \myint_{\Omega} -\nabla v \cdot \begin{pmatrix} w_{yy} & -w_{xy} \\ -w_{yx} & w_{xx} \end{pmatrix} \nabla u + \myint_{\partial \Omega} v \begin{pmatrix} w_{yy} & -w_{xy} \\ -w_{yx} & w_{xx} \end{pmatrix} \nabla u \cdot n 
%	-  \myint_\Omega v \begin{pmatrix} w_{yyx}-w_{yxy} \\ -w_{xyx} + w_{xxy} \end{pmatrix} \cdot \nabla u
%\]

\[
	a(v,\varphi) = \myint_{\Omega} \nabla \varphi \cdot \cof(D^2 w) \nabla v - \myint_{\partial \Omega} \varphi\cof(D^2 w) \nabla v \cdot n 
	+  \myint_\Omega \varphi (\nabla \cdot \cof(D^2 w)) \cdot \nabla v
\]

%Sei eine shape-regular Triangulation $\mathcal{T}_h$ mit Eckenmenge $\mathcal{E}_h$ von $\Omega$ gegeben. Wie üblich sei $\mathcal E _h $ disjunkt in die beiden Teilmengen $\mathcal{E}_h^i$, die inneren Kanten und $\mathcal{} 

Let $\mathcal{T}_h$ be shape regular, uniform triangulation of $\Omega$ and denote its edge set by $\mathcal{E}_h$. We divide $\mathcal E _h $ into the disjoint subsets $\mathcal{E}_h^i$ which contains the inner edges and $\mathcal{E}_h^b$ which contains the boundary edges. \\
In addition let $\mathcal V_h$ be the finite element space of piecewise polynomials, namely
\[
	\mathcal V_h = \{ f \in H^1(\Omega); f \arrowvert_T \textnormal{ is a polynomial of (total) degree } \leq k \textnormal{ for every } t \in \mathcal T_h\}
\]
Futhermore let $w$ be $\mathcal V_h$ such that we can define the Hessian of $w$ on every element and let us assume $\varphi \in H^2(\Omega; \triang) \cap H^1(\Omega)$.
 Decomposing our integral in the contributions of each element we gain

\begin{eqnarray*}
a(v, \varphi) &=& \sum_{T \in \triang} \myint_T \nabla \varphi \cdot \cof(D^2 w) \nabla v - \sum_{T \in \triang}\myint_{\partial T} \varphi \cof(D^2 w) \nabla v \cdot n \\
	&&+  \sum_{T \in \triang} \myint_T \varphi (\nabla \cdot \cof(D^2 w)) \cdot \nabla v
\end{eqnarray*}

If we take a closer look on $\nabla \cdot \cof(D^2 w) = \left( w_{yyx}-w_{yxy}, -w_{xyx} + w_{xxy} \right)^t$ we discover that it equals zero because of Schwarz' theorem. Thus, the last term of $a$ vanishes.

Since two adjacent elements share one edge with opposite normal vectors we can rewrite the middle term by

\begin{eqnarray*}
&\sum\limits_{T \in \triang}\myint_{\partial T} \varphi \cof(D^2 w) \nabla v \cdot n \\
= &\sum\limits_{e \in \edgesi}\myint_{e} \left( \varphi^+ \cof(D^2 w)^+ \nabla v^+ \cdot n - \varphi^- \cof(D^2 w)^- \nabla v^- \cdot n \right) \\
& + \sum\limits_{e \in \edgesb}\myint_{e} \varphi \cof(D^2 w) \nabla v \cdot n
\end{eqnarray*}

where $x^\pm $ is $x$ evaluated in one of the two adjadecent elements $T^\pm$. With the formula $ac-bd = \frac 1 2 (a+b)(c-d) + \frac 1 2 (a-b)(c+d)$ we get
\begin{eqnarray*}
	&&\varphi^+ \cof(D^2 w)^+ \nabla v^+ \cdot n - \varphi^- \cof(D^2 w)^- \nabla v^- \cdot n \\
	= && \frac 1 2 \left(\varphi^+ + \varphi^- \right) \left(\cof(D^2 w)^+ \nabla v^+ \cdot n - \cof(D^2 w)^- \nabla v^- \cdot n \right) \\
  &+ & \frac 1 2 \left(\varphi^+ - \varphi^- \right) \left(\cof(D^2 w)^+ \nabla v^+ \cdot n + \cof(D^2 w)^- \nabla v^- \cdot n \right) \\
  = &&  \{\{\varphi\}\} \llbracket \cof(D^2 w) \nabla v \cdot n \rrbracket + \llbracket\varphi\rrbracket \{\{ \cof(D^2 w) \nabla v \cdot n \}\}
\end{eqnarray*}

Therefore the weak formulation can be written as $a(v,\varphi) = l(\varphi)$ with 

\begin{eqnarray*}
 &a(v, \varphi) = & \sum\limits_{T \in \triang} \myint_T \nabla \varphi \cdot \left(\cof(D^2 w) \nabla v\right) \\
	& &- \sum\limits_{e \in \edgesi}\myint_{e} \left( \{\{\varphi\}\} \llbracket \cof(D^2 w) \nabla v \cdot n \rrbracket + \llbracket\varphi\rrbracket \{\{ \cof(D^2 w) \nabla v \cdot n \}\} \right)\\
&& - \sum\limits_{e \in \edgesb}\myint_{e} \varphi \cof(D^2 w) \nabla v \cdot n
\end{eqnarray*}
and
\[
l(\varphi) = \sum_{T \in \triang} \int_T -2v f
\]

Due to the smoothness of $u$ and $w$ we can neglect the jump in $\cof(D^2 w) \nabla u$
\begin{eqnarray*}
 &a(v, \varphi) = & \sum\limits_{T \in \triang} \myint_T \nabla \varphi \cdot \left(\cof(D^2 w) \nabla v\right) %\\
	- \sum\limits_{e \in \edgesi}
	%\myint_{e} \left( \{\{\varphi\}\} \llbracket \cof(D^2 w) \nabla v \cdot n \rrbracket + 
	\llbracket\varphi\rrbracket \{\{ \cof(D^2 w) \nabla v \cdot n \}\} %\right)
	\\
&& - \sum\limits_{e \in \edgesb}\myint_{e} \varphi \cof(D^2 w) \nabla v \cdot n
\end{eqnarray*}

Symmetrising our bilinear form $a$ by adding terms on both sides we get
\begin{eqnarray*}
 &a_S(v, \varphi) = &\sum\limits_{T \in \triang} \myint_T \nabla \varphi \cdot \cof(D^2 w) \nabla v \\
 & &-\sum\limits_{e \in \edgesi}\myint_{e} \llbracket\varphi\rrbracket \{\{ \cof(D^2 w) \nabla v \cdot n \}\} 
 - \sum\limits_{e \in \edgesi}\myint_{e} \llbracket v\rrbracket \{\{ \cof(D^2 w) \nabla \varphi \cdot n \}\} \\  
 && - \sum\limits_{e \in \edgesb}\myint_{e} \varphi \cof(D^2 w) \nabla v \cdot n 
    - \sum\limits_{e \in \edgesb}\myint_{e} v \cof(D^2 w) \nabla \varphi \cdot n
\end{eqnarray*}
and 
\begin{eqnarray*}
	l(\varphi) = &-& \sum\limits_{T \in \triang} \myint_T \varphi 2f \\
	 				&-&\sum\limits_{e \in \edgesb}\myint_{e} u_0 \cof(D^2 w) \nabla \varphi \cdot n 
\end{eqnarray*} 

%and $f$
%\begin{eqnarray*}
%	f_S(v,\varphi) = && \sum\limits_{T \in \triang} \myint_T \varphi f \\
%	 				&+ &\sum\limits_{e \in \edgesb}\myint_{e} \varphi \cof(D^2 w) \nabla v \cdot n \\
% &+ &\sum\limits_{e \in \edgesi}\myint_{e} v \llbracket \cof(D^2 w) \nabla \varphi \cdot n\rrbracket \\
%	&-  &\sum\limits_{T \in \triang} \myint_T v (\nabla \cdot \cof(D^2w)) \cdot \nabla \varphi \\
%\end{eqnarray*} 

To enforce stability we enforce the following penalty terms [TICAM report 3.2.2.]
\begin{eqnarray*}
	J^\sigma(\varphi, v) = \sum\limits_{e \in \edges} \myint_e \frac \sigma {|e|} \llbracket \varphi \rrbracket \llbracket v \rrbracket \textnormal{ and } 	J^\sigma_0(\varphi, v) = \sum\limits_{e \in \edgesb} \myint_e \frac \sigma {|e|} \varphi u_0  
\end{eqnarray*}

Thus, we end up with the problem finding $v \in V_h \subset H^1(\Omega)$ such that
\[
	a_S(\phi,v) + J^\sigma(\varphi,v) = f(\varphi) + J^\sigma_0(\varphi)
\] 
$  \textnormal{for all } \varphi \in H^2(\Omega; \triang) \cap H^1(\Omega)$
\end{document}
